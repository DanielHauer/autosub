\documentclass[a4paper,12pt]{article}
\usepackage{a4wide}
\usepackage{pst-circ}
\usepackage{tikz}
\usetikzlibrary{calc}

 \usepackage[ngerman]{babel}

\begin{document}
\pagestyle{empty}
\setlength{\parindent}{0em} 

% c for custom
\newcommand{\cand}[9]{\logicand[ninputs=4,iec=true,bubblesize=0.1,inputa=#1,inputb=#2,inputc=#3,inputd=#4, ,invertinputa=#5,invertinputb=#6,invertinputc=#7,invertinputd=#8,invertoutput=#9]}

\newcommand{\cor}[9]{\logicor[ninputs=4,iec=true,bubblesize=0.1,inputa=#1,inputb=#2,inputc=#3,inputd=#4, ,invertinputa=#5,invertinputb=#6,invertinputc=#7,invertinputd=#8,invertoutput=#9]}

\newcommand{\cxor}[9]{\logicxor[ninputs=4,iec=true,bubblesize=0.1,inputa=#1,inputb=#2,inputc=#3,inputd=#4, ,invertinputa=#5,invertinputb=#6,invertinputc=#7,invertinputd=#8,invertoutput=#9]}


\section*{Basic Gates}
Ihre Aufgabe ist es, das Verhalten einer Entity  namens "`gates"' zu programmieren. Die Entity ist in der angeh\"angten Datei "`gates.vhdl"' deklariert und hat folgende Eigenschaften:

\begin{itemize}
\item Eing\"ange: A, B, C, D vom Typ std\_logic
\item Ausg\"ange: O vom Typ std\_logic
\end{itemize}

\begin{center}
\begin{tikzpicture}
\draw node [draw,rectangle, minimum height=25mm, minimum width=35mm,rounded corners=2mm,thick](entity){};
\draw[->,thick] ($ (entity.west)-(10mm,7.5mm)$) -- ($ (entity.west) - (0mm,7.5mm)$);
\draw node at ($ (entity.west)-(12mm,7.5mm)$){D};
\draw[->,thick] ($ (entity.west)-(10mm,2.5mm)$) -- ($ (entity.west) - (0mm,2.5mm)$);
\draw node at ($ (entity.west)-(12mm,2.5mm)$){C};
\draw[->,thick] ($ (entity.west)-(10mm,-2.5mm)$) -- ($ (entity.west) - (0mm,-2.5mm)$);
\draw node at ($ (entity.west)-(12mm,-2.5mm)$){B};
\draw[->,thick] ($ (entity.west)-(10mm,-7.5mm)$) -- ($ (entity.west) - (0mm,-7.5mm)$);
\draw node at ($ (entity.west)-(12mm,-7.5mm)$){A};

\draw[->,thick] (entity.east) -- ($ (entity.east) + (10mm,0)$);;
\draw node at ($ (entity.east) + (12mm,0)$){O};

\draw node at ($ (entity) - (0,16mm)$){gates};

\end{tikzpicture}
\end{center}

Ver\"andern Sie die Datei  "`gates.vhdl"' nicht!
\\

Die Entity "`gates"' soll sich entsprechend dem folgenden Netzwerk verhalten:

\vspace{0.3cm}

\begin{center}
\psset{unit=0.9}
\begin{pspicture}(12,13)
%\psgrid
\psset{dotsize=0.15}

%input gates
%<b> =true/false
%       inputs enabled(a-d)  inputs negates(a-d)  output negated 
%\<type> {<b>}{<b>}{<b>}{<b>} {<b>}{<b>}{<b>}{<b>} {<b>}(2,1){G0}
%\<type> {<b>}{<b>}{<b>}{<b>} {<b>}{<b>}{<b>}{<b>} {<b>}(2,4){G1}
%\<type> {<b>}{<b>}{<b>}{<b>} {<b>}{<b>}{<b>}{<b>} {<b>}(2,7){G2}
%\<type> {<b>}{<b>}{<b>}{<b>} {<b>}{<b>}{<b>}{<b>} {<b>}(2,10){G3}

%\<type> {<b>}{<b>}{<b>}{<b>} {<b>}{<b>}{<b>}{<b>} {<b>}(7,5.5){G4}

{{G0}} {{ IE0  }}{{ IE1  }}{{ IE2  }}{{ IE3  }} {{ IN0  }}{{ IN1  }}{{ IN2  }}{{ IN3  }} {{ ON0 }}(2,1){G0}
{{G1}} {{ IE4  }}{{ IE5  }}{{ IE6  }}{{ IE7  }} {{ IN4  }}{{ IN5  }}{{ IN6  }}{{ IN7  }} {{ ON1 }}(2,4){G1}
{{G2}} {{ IE8  }}{{ IE9  }}{{ IE10 }}{{ IE11 }} {{ IN8  }}{{ IN9  }}{{ IN10 }}{{ IN11 }} {{ ON2 }}(2,7){G2}
{{G3}} {{ IE12 }}{{ IE13 }}{{ IE14 }}{{ IE15 }} {{ IN12 }}{{ IN13 }}{{ IN14 }}{{ IN15 }} {{ ON3 }}(2,10){G3}

%output gate
{{G4}} {true}{true}{true}{true} {{ IN19 }}{{ IN18 }}{{ IN17 }}{{ IN16 }} {{ ON4 }}(7,5.5){G4}


%input lines and text
\psline{-}(0.4,12.4)(0.4,1.25)
\psline{-}(0.8,12.4)(0.8,1.25)
\psline{-}(1.2,12.4)(1.2,1.25)
\psline{-}(1.6,12.4)(1.6,1.25)
\rput(0.4,12.65){A}
\rput(0.8,12.65){B}
\rput(1.2,12.65){C}
\rput(1.6,12.65){D}

%output
\rput(10.8,6.5){O}

%lines input->gates, go=gateoffset 
%\psline{*-}(1.6,go+0.25)(2,go+0.25)   %d
%\psline{*-}(1.2,go+0.75)(2,go+0.75)   %c
%\psline{*-}(0.8,go+1.25)(2.1,go+1.25) %b
%\psline{*-}(0.4,go+1.75)(2.1,go+1.75) %a

%G0 go=1 
{{GI0}}   \psline{*-}(0.4,2.75)(2.1,2.75)%a
{{GI1}}   \psline{*-}(0.8,2.25)(2.1,2.25)%b
{{GI2}}   \psline{*-}(1.2,1.75)(2,1.75)%c
{{GI3}}   \psline{*-}(1.6,1.25)(2,1.25)%d

%G1 go=4
{{GI4}}  \psline{*-}(0.4,5.75)(2.1,5.75) %a
{{GI5}}   \psline{*-}(0.8,5.25)(2.1,5.25) %b
{{GI6}}   \psline{*-}(1.2,4.75)(2,4.75)   %c
{{GI7}}   \psline{*-}(1.6,4.25)(2,4.25)   %d

%G2 g0=7
{{GI8}}   \psline{*-}(0.4,8.75)(2.1,8.75) %a
{{GI9}}   \psline{*-}(0.8,8.25)(2.1,8.25) %b
{{GI10}}  \psline{*-}(1.2,7.75)(2,7.75)   %c
{{GI11}}  \psline{*-}(1.6,7.25)(2,7.25)   %d

%G3 go=10\\
{{GI12}} \psline{*-}(0.4,11.75)(2.1,11.75) %a
{{GI13}} \psline{*-}(0.8,11.25)(2.1,11.25) %b
{{GI14}} \psline{*-}(1.2,10.75)(2,10.75)   %c
{{GI15}} \psline{*-}(1.6,10.25)(2,10.25)   %d


%G0 to G4
\psline{-}(5.25,2)(7,2)
\psline{-}(7,2)(7,5.75) 

%G1 to G4
\psline{-}(5.5,5)(5.5,6.25)
\psline{-}(5.5,6.25)(7,6.25)

%G2 to G4
\psline{-}(5.5,6.75)(5.5,8)
\psline{-}(5.5,6.75)(7,6.75)


%G3 to G4
\psline{-}(5.25,11)(7,11) 
\psline{-}(7,11)(7,7.25) 

\end{pspicture}
\end{center}

Dieses Verhalten muss in der angeh\"angten Daten "`gates\_beh.vhdl"' programmiert werden.
\\

Sie k\"onnen die in VHDL vordefinierten Entities zur L\"osung der Aufgabe verwenden. Ben\"utzen Sie dazu die vorhandenen IEEE 1164 Logikgatter, um das geforderte Netzwerk zu konstruieren. Es stehen Ihnen 6 verschiedene Gatter-Entities zur Verf\"ugung: AND[N], NAND[N], OR[N], NOR[N], XOR[N], XNOR[N], wobei [N] der Anzahl an Eing\"angen entspricht. Die Eing\"ange sind mit I1..I[N] beschriftet und der einzelne Ausgang mit O.
\\

 Ein Beispiel: Ein NAND Gatter mit 3 Eing\"angen hei"st NAND3 und besitzt die Eing\"ange I1, I2 und I3 und den Ausgang O. Die ben\"otigten Programmpakete zur Benutzung dieser Entities sind bereits in der Datei  "`gates\_beh.vhdl"' importiert.
\\

Um Ihre L\"osung abzugeben, senden Sie ein E-Mail mit dem Betreff "'Result Task {{ TASKNR }}"' und Ihrer Datei "`gates\_beh.vhdl"'  an {{ SUBMISSIONEMAIL }}. 

\vspace{0.3cm}

Viel Erfolg und m\"oge die Macht mit Ihnen sein!

\end{document}
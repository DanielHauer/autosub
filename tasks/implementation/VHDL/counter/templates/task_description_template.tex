\documentclass[a4paper,12pt]{article}
\usepackage{a4wide}
\usepackage{tikz}
\usetikzlibrary{calc}
\usepackage{hyperref}

\begin{document}
\pagestyle{empty}
\setlength{\parindent}{0em}
\section*{\noindent Counter }
Your task is to program the behavior of an entity called ``counter". This entity is declared in the attached file ``counter.vhdl" and has the following properties:

\begin{itemize}
	\item Input: CLK with type std\_logic
	%%enable_property_desc
	%%sync_property_desc
	%%async_property_desc
	%%input_property_desc
	%%overflow_property_desc
	\item Output: Output with type std\_logic\_vector of length %%counter_width

\begin{center}
\begin{tikzpicture}
\draw node [draw,rectangle, minimum height=%%minimum_height mm, minimum width=35mm,rounded corners=2mm,thick](entity){};
%%inputs_tikz
%%outputs_tikz
\draw node at ($ (entity) - (0,0mm)$){ counter };
\end{tikzpicture}
\end{center}

\end{itemize}

%\begin{center}

Do not change the file ``counter.vhdl".\\

The ``counter" entity shall increment the Output vector on %%every_a rising edge of the CLK signal. The initial value of the Output, before the first rising edge of the CLK signal, shall be ``%%init_value_padded". When the Sync%%sync_variation signal is set to `1' then the Output vector shall be set to %%sync_text at the rising edge of the CLK signal. When the Async%%async_variation signal is set to `1' then the Output vector shall be set to %%async_text immediately. %%Enable_Overflow_text \\

This behavior has to be programmed in the attached file ``counter\_beh.vhdl".\\


To turn in your solution write an email to %%SUBMISSIONEMAIL with Subject ``Result Task %%TASKNR" and attach your behavior file ``counter\_beh.vhdl".

\vspace{0.7cm}
Good Luck and May the Force be with you.


\end{document}
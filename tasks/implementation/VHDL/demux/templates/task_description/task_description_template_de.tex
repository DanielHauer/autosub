\documentclass[a4paper,12pt]{article}
\usepackage{a4wide}
\usepackage{tikz}
\usetikzlibrary{calc}
\usepackage{hyperref}

\usepackage[ngerman]{babel}

\newcommand\Tstrut{\rule{0pt}{2.6ex}}       % "top" strut

\begin{document}
\pagestyle{empty}
\setlength{\parindent}{0em}
\section*{\noindent Demux }
Ihre Aufgabe ist es, das Verhalten einer Entity  namens "`demux"' zu programmieren. Die Entity ist in der angeh\"angten Datei "`demux.vhdl"' deklariert und hat folgende Eigenschaften:

\begin{itemize}
	\item Eingang: IN1 vom Typ std\_logic\_vector mit einer L\"ange von {{IN1_width}}
	\item Eingang: SEL vom Typ std\_logic\_vector mit einer L\"ange von {{SEL_width}}
	\item Ausg\"ange: {{outputs_comma}} vom Typ std\_logic\_vector mit einer L\"ange von {{IN1_width}}
\end{itemize}

\begin{center}
\begin{tikzpicture}
\draw node [draw,rectangle, minimum height={{minimum_height}} mm, minimum width=35mm,rounded corners=2mm,thick](entity){};

\draw[->] ($ (entity.west)+(-10mm,3mm)$) -- ($ (entity.west) + (0mm,3mm)$);
\draw[anchor=east] node at ($ (entity.west)+(-9mm,3mm)$){ IN1 };
\draw[->] ($ (entity.west)+(-10mm,-3mm)$) -- ($ (entity.west) + (0mm,-3mm)$);
\draw[anchor=east] node at ($ (entity.west)+(-9mm,-3mm)$){ SEL };

{{outputs_entity}}

% \draw[->] ($ (entity.east) + (0mm,6.0mm)$) -- ($ (entity.east) + (10mm,6.0mm)$);
% \draw[anchor=west] node at ($ (entity.east) + (9mm,6.0mm)$){ OUT1 };

\draw node at ($ (entity) - (0,0mm)$){ demux };

\end{tikzpicture}
\end{center}

Ver\"andern Sie die Datei "`demux.vhdl" nicht!\\

Die Entity "`demux"' soll das Eingangssignal IN1 entsprechend dem Steuersignal SEL auf einen von mehreren Ausg\"angen \mbox{OUT1 -- {{num_out}}} weiterleiten. Ist der Steuereingang SEL gleich "`{{SEL_binary_zero}}"', so soll der Eingang IN1 an den Ausgang OUT1 geleitet werden. Wenn der Steuereingang SEL gleich "`{{SEL_binary_one}}"' ist, dann soll der Eingang IN1 an den Ausgang OUT2 geleitet werden. Dieses Verhalten soll sich entsprechend Tabelle~1 fortsetzen. Solange ein Ausgang nicht ausgew\"ahlt ist, sollen alle seine Ausgangs-Bits auf 0 gesetzt sein. {{SEL_max_greater_num_out}}

\begin{table}[h!]
\centering
    \begin{tabular}{|c| {{num_c}} |} \hline \Tstrut
		SEL & {{SEL_possible}}  \\ \hline \Tstrut
		aktiver Ausgang & {{OUT_selected}} \\
		\hline
    \end{tabular}
    \caption{aktiver Ausgang in Abh\"angigkeit von SEL}
    \label{tab:SELoutput}
\end{table}

Programmieren Sie dieses Verhalten in der angeh\"angten Datei "`demux\_beh.vhdl"'.\\

Um Ihre L\"osung abzugeben, senden Sie ein E-Mail mit dem Betreff "`Result Task {{TASKNR}}"' und Ihrer Datei "`demux\_beh.vhdl"'  an {{SUBMISSIONEMAIL}}.

\vspace{0.7cm}
Viel Erfolg und m\"oge die Macht mit Ihnen sein.

\end{document}

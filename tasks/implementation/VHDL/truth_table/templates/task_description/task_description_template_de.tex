\documentclass[a4paper,12pt]{article}
\usepackage{a4wide}
\usepackage{tikz}
\usetikzlibrary{calc}


\usepackage[ngerman]{babel}

\begin{document}
\pagestyle{empty}
\setlength{\parindent}{0em} 
\section*{Truth Table}

Ihre Aufgabe ist es, das Verhalten einer Entity  namens "`truth\_table"' (Wahrheitstabelle) zu programmieren. Die Entity ist in der angeh\"angten Datei "`truth\_table.vhdl"' deklariert und hat folgende Eigenschaften:
\begin{itemize}
\item Eing\"ange:  A, B, C, D vom Typ  std\_logic
\item Ausgang: O vom Typ std\_logic
\end{itemize}

\begin{center}
\begin{tikzpicture}
\draw node [draw,rectangle, minimum height=25mm, minimum width=35mm,rounded corners=2mm,thick](entity){};
\draw[->,thick] ($ (entity.west)-(10mm,7.5mm)$) -- ($ (entity.west) - (0mm,7.5mm)$);
\draw node at ($ (entity.west)-(12mm,7.5mm)$){D};
\draw[->,thick] ($ (entity.west)-(10mm,2.5mm)$) -- ($ (entity.west) - (0mm,2.5mm)$);
\draw node at ($ (entity.west)-(12mm,2.5mm)$){C};
\draw[->,thick] ($ (entity.west)-(10mm,-2.5mm)$) -- ($ (entity.west) - (0mm,-2.5mm)$);
\draw node at ($ (entity.west)-(12mm,-2.5mm)$){B};
\draw[->,thick] ($ (entity.west)-(10mm,-7.5mm)$) -- ($ (entity.west) - (0mm,-7.5mm)$);
\draw node at ($ (entity.west)-(12mm,-7.5mm)$){A};

\draw[->,thick] (entity.east) -- ($ (entity.east) + (10mm,0)$);;
\draw node at ($ (entity.east) + (12mm,0)$){O};

\draw node at ($ (entity) - (0,16mm)$){truth\_table};

\end{tikzpicture}
\end{center}

Ver\"andern Sie die Datei "`truth\_table.vhdl"' nicht!\\

Die Entity "`truth\_table"' soll sich entsprechend der folgenden Wahrheits\-tabelle verhalten:

\vspace{0.3cm}
\begin{center}
\begin{tabular}{||l | l | l | l || l||} 
	\hline
	D & C & B & A & O\\ [0.5ex] 
	\hline\hline
	0&0&0&0& {{ O0 }}
	\\
	\hline
	0&0&0&1& {{ O1 }}
	\\
	\hline
	0&0&1&0& {{ O2 }}
	\\
	\hline
	0&0&1&1& {{ O3 }}
	\\
	\hline\hline
	0&1&0&0& {{ O4 }}
	\\
	\hline
	0&1&0&1& {{ O5 }}
	\\
	\hline
	0&1&1&0& {{ O6 }}
	\\
	\hline
	0&1&1&1& {{ O7 }}
	\\
	\hline\hline
	1&0&0&0& {{ O8 }}
	\\
	\hline
	1&0&0&1& {{ O9 }}
	\\
	\hline
	1&0&1&0& {{ O10 }}
	\\
	\hline
	1&0&1&1& {{ O11 }}
	\\
	\hline\hline
	1&1&0&0& {{ O12 }}
	\\
	\hline
	1&1&0&1& {{ O13 }}
	\\
	\hline
	1&1&1&0& {{ O14 }}
	\\
	\hline
	1&1&1&1& {{ O15 }}
	\\
	\hline\hline
\end{tabular}
\end{center}

\vspace{0.3cm}

Dieses Verhalten muss in der angeh\"angten Datei "`truth\_table\_beh.vhdl"' programmiert werden. Dazu wird zu Beginn eine Vereinfachung etwa mittels Karnaugh-Veitch-Diagramm (KV Diagramm) empfohlen.

Um Ihre L\"osung abzugeben, senden Sie ein E-Mail mit dem Betreff "`Result Task {{TASKNR}}"' und Ihrer Datei "`truth\_table\_beh.vhdl"' an {{SUBMISSIONEMAIL}}. 

\vspace{0.7cm}
Viel Erfolg und m\"oge die Macht mit Ihnen sein.

\end{document}

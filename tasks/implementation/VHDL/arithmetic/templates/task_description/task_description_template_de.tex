\documentclass[a4paper,12pt]{article}
\usepackage{a4wide}
\usepackage{tikz}
\usetikzlibrary{calc}
\usepackage{hyperref}

\usepackage[ngerman]{babel}

\begin{document}
\pagestyle{empty}
\setlength{\parindent}{0em}
\section*{Arithmetic}

Ihre Aufgabe ist es, das Verhalten einer Entity  namens "`arithmetic"' zu programmieren. Die Entity ist in der angeh\"angten Datei "`arithmetic.vhdl"' deklariert und hat folgende Eigenschaften:
\begin{itemize}
\item Eing\"ange:  I1, I2 vom Typ std\_logic\_vector
\item Ausg\"ange: O vom Typ std\_logic\_vector, V vom Typ std\_logic, C vom Typ std\_logic, VALID vom Typ std\_logic
\end{itemize}

\begin{center}
\begin{tikzpicture}
\draw node [draw,rectangle, minimum height=25mm, minimum width=35mm,rounded corners=2mm,thick](entity){};
\draw[->,thick] ($ (entity.west)-(10mm,5mm)$) -- ($ (entity.west) - (0mm,5mm)$);
\draw node at ($ (entity.west)-(12mm,5mm)$){I2};
\draw[->,thick] ($ (entity.west)-(10mm,-5mm)$) -- ($ (entity.west) - (0mm,-5mm)$);
\draw node at ($ (entity.west)-(12mm,-5mm)$){I1};

\draw[->,thick] ($ (entity.east) + (0mm,7.5mm)$) -- ($ (entity.east) + (10mm,7.5mm)$);
\draw node at ($ (entity.east) + (12mm,7.5mm)$){O};
\draw[->,thick] ($ (entity.east) + (0mm,2.5mm)$) -- ($ (entity.east) + (10mm,2.5mm)$);
\draw node at ($ (entity.east) + (12mm,2.5mm)$){V};
\draw[->,thick] ($ (entity.east) + (0mm,-2.5mm)$) -- ($ (entity.east) + (10mm,-2.5mm)$);
\draw node at ($ (entity.east) + (12mm,-2.5mm)$){C};
\draw[->,thick] ($ (entity.east) + (0mm,-7.5mm)$) -- ($ (entity.east) + (10mm,-7.5mm)$);
\draw node at ($ (entity.east) + (17mm,-7.5mm)$){VALID};

\draw node at ($ (entity) - (0,15mm)$){arithmetic};

\end{tikzpicture}
\end{center}

Ver\"andern sie die Datei "`arithmetic.vhdl"' nicht!
\\

Die Entity soll folgende arithmetische Funktionen implementieren:
\begin{itemize}
\item {{OPERATION_NAME}} I1 {{OPERATION_SIGN}} I2
\item Eingangsoperand (I1): {{I1_WIDTH}} Bit, {{OPERAND_STYLE}}
\item Eingangsoperand 2 (I2): {{I2_WIDTH}} Bit, {{OPERAND_STYLE}}
\item Ausgang (O): {{O_WIDTH}} Bit, {{OPERAND_STYLE}}
\item Overflow (V) und Carry Flag (C) entsprechend gesetzt
\item Valid Flag (VALID): Gibt an, ob  die berechnete L\"osung g\"ultig ist oder nicht.
\end{itemize}

Dieses Verhalten soll von Ihnen in der Datei "`arithmetic\_beh.vhdl"' programmiert werden.
\\

\textbf{Anmerkung:} Die ALU soll mit einer Bitl\"ange von {{O_WIDTH}} implementiert werden. Daher kann eine Vorzeichenerweiterung der Operanden notwendig sein.

\newpage
\subsection*{Zusammenfassung: Carry und Overflow Flag}

\textbf{Carry Flag}\\
Das Carry Flag wird in der bin\"aren Mathematik unter zwei Bedingungen gesetzt (='1'):
\begin{enumerate}
\item Das Carry Flag wird gesetzt, wenn bei der Addition von zwei Zahlen das MSB (h\"ochstwertigste Bit) des Ergebnisses aus den verf\"ugbaren Bit-Stellen "`herausgeschoben"' wird.

   z.B. 1111 + 0001 = 0000 (Carry Flag ist '1')

\item Das Carry (Borrow) Flag wird ebenfalls bei der Subtraktion zweier Zahlen gesetzt, wenn dabei an der Stelle des MSB ein Bit "`ausgeborgt"' werden muss.

   z.B. 0000 - 0001 = 1111 (Carry Flag ist '1')
\end{enumerate}

Ansonsten ist das Carry Flag immer nicht gesetzt (='0').

\begin{itemize}
\item z.B. 0111 + 0001 = 1000 (Carry Flag ist '0')
\item z.B. 1000 - 0001 = 0111 (Carry Flag ist '0')
\end{itemize}

%CHANGED: omit 1s complement
%Im Einerkomplement wird das Carry Flag vor einer m\"oglichen R\"uckf\"uhrung des \"Ubertrags ("`end around"') gesetzt.

In vorzeichenloser Arithmetik zeigt ein gesetzes Carry Flag, dass das Ergebis falsch ist.

In vorzeichenbehafteter Arithmetik liefert das Carry Flag keine relevanten Informationen.
\\

\textbf{Overflow Flag}\\
Das Overflow Flag wird in der bin\"aren Mathematik unter folgenden Bedingungen gesetzt (='1'):

\begin{enumerate}
\item Wenn die Summe zweier Zahlen, welche jeweils kein gesetzes Vorzeichenbit aufweisen, zu einem Ergebnis mit einem gesetzen Vorzeichenbit f\"uhrt:

   z.B. 0100 + 0100 = 1000 (Overflow Flag ist '1')

\item Wenn die Summe zweier Zahlen, welche jeweils ein gesetzes Vorzeichenbit aufweisen, zu einem Ergebnis mit einem nicht gesetzten Vorzeichenbit f\"uhrt:

   z.B. 1000 + 1000 = 0000 (Overflow Flag ist '1')

\item Wenn die Subtraktion einer Zahl mit gesetzem Vorzeichenbit von einer Zahl mit nicht gesetzem Vorzeichenbit zu einem Ergebnis mit gesetzem Vorzeichenbit f\"uhrt:

   z.B. 0111 - 1001 = 1110 (Overflow Flag ist '1')

\item Wenn die Subtraktion einer Zahl mit nicht gesetzem Vorzeichenbit von einer Zahl mit gesetzem Vorzeichenbit zu einem Ergebnis mit nicht gesetzem Vorzeichenbit f\"uhrt:

   z.B. 1001 - 0111 = 0010 (Overflow Flag ist '1')
\end{enumerate}

Ansonsten ist das Overflow Flag immer nicht gesetzt (='0').

\begin{itemize}
\item z.B. 0100 + 0001 = 0101 (Overflow Flag ist '0')
\item z.B. 0110 + 1001 = 1111 (Overflow Flag ist '0')
\item z.B. 1000 + 0001 = 1001 (Overflow Flag ist '0')
\item z.B. 1100 + 1100 = 1000 (Overflow Flag ist '0')
\end{itemize}

%CHANGED: omit 1s complement
%Im Einerkomplement wird das Overflow Flag nach einer m\"oglichen R\"uckf\"uhrung des \"Ubertrags ("`end around"') gesetzt.

Bei vorzeichenbehafteter Arithmetik zeigt ein gesetztes Overflow Flag, dass das Ergebnis falsch ist. 

Bei der vorzeichenlosen Arithmetik liefert das Overflow Flag keine relevanten Informationen.

\rule{16cm}{0.4pt}

\vspace{0.3cm}

Tipp: Verwenden Sie das "`ieee.numeric\_std"' Package f\"ur die Implementierung der arithmetischen Operation.
\\

Um Ihre L\"osung abzugeben, senden Sie ein E-Mail mit dem Betreff "`Result Task {{ TASKNR }}"' und Ihrer Datei "`arithmetic\_beh.vhdl"' an {{ SUBMISSIONEMAIL }}.

\vspace{0.7cm}

Viel Erfolg und m\"oge die Macht mit Ihnen sein.
\end{document}

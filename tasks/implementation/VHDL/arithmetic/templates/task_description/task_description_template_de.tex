\documentclass[a4paper,12pt]{article}
\usepackage{a4wide}
\usepackage{tikz}
\usetikzlibrary{calc}
\usepackage{hyperref}

\usepackage[ngerman]{babel}

\begin{document}
\pagestyle{empty}
\setlength{\parindent}{0em}
\section*{Arithmetic}

Ihre Aufgabe ist es, das Verhalten einer Entity  namens "`arithmetic"' zu programmieren. Die Entity ist in der angeh\"angten Datei "`arithmetic.vhdl"' deklariert und hat folgende Eigenschaften:
\begin{itemize}
\item Eing\"ange:  I1, I2 vom Typ std\_logic\_vector
\item Ausg\"ange: O vom Typ std\_logic\_vector, V vom Typ std\_logic, C vom Typ std\_logic, VALID vom Typ std\_logic
\end{itemize}

\begin{center}
\begin{tikzpicture}
\draw node [draw,rectangle, minimum height=25mm, minimum width=35mm,rounded corners=2mm,thick](entity){};
\draw[->,thick] ($ (entity.west)-(10mm,5mm)$) -- ($ (entity.west) - (0mm,5mm)$);
\draw node at ($ (entity.west)-(12mm,5mm)$){I2};
\draw[->,thick] ($ (entity.west)-(10mm,-5mm)$) -- ($ (entity.west) - (0mm,-5mm)$);
\draw node at ($ (entity.west)-(12mm,-5mm)$){I1};

\draw[->,thick] ($ (entity.east) + (0mm,7.5mm)$) -- ($ (entity.east) + (10mm,7.5mm)$);
\draw node at ($ (entity.east) + (12mm,7.5mm)$){O};
\draw[->,thick] ($ (entity.east) + (0mm,2.5mm)$) -- ($ (entity.east) + (10mm,2.5mm)$);
\draw node at ($ (entity.east) + (12mm,2.5mm)$){V};
\draw[->,thick] ($ (entity.east) + (0mm,-2.5mm)$) -- ($ (entity.east) + (10mm,-2.5mm)$);
\draw node at ($ (entity.east) + (12mm,-2.5mm)$){C};
\draw[->,thick] ($ (entity.east) + (0mm,-7.5mm)$) -- ($ (entity.east) + (10mm,-7.5mm)$);
\draw node at ($ (entity.east) + (17mm,-7.5mm)$){VALID};

\draw node at ($ (entity) - (0,15mm)$){arithmetic};

\end{tikzpicture}
\end{center}

Ver\"andern sie die Datei "`arithmetic.vhdl"' nicht!
\\

Die Entity soll folgende arithmetische Funktionen implementieren:
\begin{itemize}
\item {{OPERATION_NAME}} I1 {{OPERATION_SIGN}} I2
\item Eingangsoperand (I1): {{I1_WIDTH}} Bit, {{OPERAND_STYLE}}
\item Eingangsoperand 2 (I2): {{I2_WIDTH}} Bit, {{OPERAND_STYLE}}
\item Ausgang (O): {{O_WIDTH}} Bit, {{OPERAND_STYLE}}
\item Overflow (V) und Carry Flag (C) entsprechend gesetzt
\item Valid Flag (VALID): Gibt an, ob  die berechnete L\"osung g\"ultig ist oder nicht.
\end{itemize}

Dieses Verhalten soll von Ihnen in der Datei "`arithmetic\_beh.vhdl"' programmiert werden.


\textbf{Anmerkung:} Die ALU soll mit einer Bitl\"ange von {{O_WIDTH}} implementiert werden. Daher kann eine Vorzeichenerweiterung der Operanden notwendig sein.

\rule{16cm}{0.4pt}\par
\subsection*{Zusammenfassung: Carry und Overflow Flag}

\textbf{Carry Flag}\\
Das Carry Flag wird in der bin\"aren Mathematik unter zwei Bedingungen gesetzt (='1'):
\begin{enumerate}
\item Das Carry Flag wird gesetzt, wenn bei der Addition von zwei Zahlen das MSB (h\"ochstwertigste oder "`linkeste"' Bit) des Ergebnisses aus den verf\"ugbaren Bit-Stellen "`herausgeschoben"' wird.

   1111 + 0001 = 0000 (Carry Flag ist '1')

\item Das Carry (Borrow) Flag wird ebenfalls bei der Subtraktion zweier Zahlen gesetzt, wenn dabei an der Stelle des MSB ein Bit "`ausgeborgt"' werden muss.

   0000 - 0001 = 1111 (Carry Flag ist '1')

Ansonsten ist das Carry Flag immer nicht gesetzt (='0').

\begin{itemize}
\item 0111 + 0001 = 1000 (Carry Flag ist '0')
\item 1000 - 0001 = 0111 (Carry Flag ist '0')
\end{itemize}

Im Einerkomplement wird das Carry Flag vor einer m\"oglichen R\"uckf\"uhrung des \"Ubertrags ("`end around"') gesetzt.

In vorzeichenloser Darstellung zeigt das Carry Flag Fehler an.
In vorzeichenbehafteter Darstellung liefert das Carry Flag keine relevanten Informationen.


\end{enumerate}

\textbf{Overflow Flag}\\
Das Overflow Flag wird in der bin\"aren Mathematik unter zwei Bedingungen gesetzt (='1'):
\begin{enumerate}
\item Wenn die Summe zweier Zahlen, welche jeweils kein gesetztes Vorzeichenbit aufweisen, zu einem Ergebnis mit einem gesetzten Vorzeichenbit f\"uhrt, dann wird das Overflow Flag gesetzt.

   0100 + 0100 = 1000 (Overflow Flag ist '1')

\item Wenn die Summe zweier Zahlen, welche jeweils ein gesetztes Vorzeichenbit aufweisen, zu einem Ergebnis ohne gesetztem Vorzeichenbit f\"uhrt, dann wird das Overflow Flag gesetzt.

   1000 + 1000 = 0000 (Overflow Flag ist '1')
\end{enumerate}

Ansonsten ist das Overflow Flag immer nicht gesetzt (='0').
\begin{itemize}
\item 0100 + 0001 = 0101 (Overflow Flag ist '0')
\item 0110 + 1001 = 1111 (Overflow Flag ist '0')
\item 1000 + 0001 = 1001 (Overflow Flag ist '0')
\item 1100 + 1100 = 1000 (Overflow Flag ist '0')
\end{itemize}

Im Einerkomplement wird das Overflow Flag nach einer m\"oglichen R\"uckf\"uhrung des \"Ubertrags ("`end around"') gesetzt.

Bei vorzeichenbehafteter Mathematik bedeutet ein gesetztes Overflow Flag, dass das Ergebnis falsch ist: Man hat zwei positive Zahlen addiert und ein negatives Ergebnis erhalten oder zwei negative Zahlen addiert und ein positives Ergebnis erhalten. Bei der vorzeichenlosen Arithmetik ist das Overflow Flag bedeutungslos und kann ignoriert werden.

\vspace{0.5cm}
\textit{Quelle: }\url{http://teaching.idallen.com/dat2343/10f/notes/040_overflow.txt}
\\
\rule{16cm}{0.4pt}

\vspace{0.3cm}

Tipp: Verwenden Sie das "`ieee.numeric\_std"' Package f\"ur die Implementierung der arithmetischen Operation.
\\

Um Ihre L\"osung abzugeben, senden Sie ein E-Mail mit dem Betreff "`Result Task {{ TASKNR }}"' und Ihrer Datei "`arithmetic\_beh.vhdl"'  an {{ SUBMISSIONEMAIL }}.

\vspace{0.7cm}

Viel Erfolg und m\"oge die Macht mit Ihnen sein.
\end{document}

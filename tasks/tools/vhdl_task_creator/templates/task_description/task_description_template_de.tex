\documentclass[a4paper,12pt]{article}
\usepackage{a4wide}
\usepackage{tikz}
\usetikzlibrary{calc}
\usepackage[ngerman]{babel}

\begin{document}
\pagestyle{empty}
\setlength{\parindent}{0em} 
\section*{\noindent {{ task_name }} }


Ihre Aufgabe ist es, das Verhalten einer Entity  namens "`{{ entity.name }}"' zu programmieren. Die Entity ist in der angeh\"angten Datei "`{{ entity.name }}.vhdl"' deklariert und hat folgende Eigenschaften:

\begin{itemize}

	\item Eingang: {{ input.name }} vom Typ {{ input.type }} mit L\"ange {{ input.length }} 



	\item Ausgang: {{ output.name }} vom Typ {{ output.type }} mit L\"ange {{ output.length }} 



\end{itemize}

\begin{center}
\begin{tikzpicture}
\draw node [draw,rectangle, minimum height={{ entity.minimum_height }}mm, minimum width=35mm,rounded corners=2mm,thick](entity){};


\draw[->] ($ (entity.west)+(-10mm,{{ (entity.minimum_height / 2) - (loop.index * (entity.minimum_height / (loop.length + 1)))}}mm)$) -- ($ (entity.west) + (0mm,{{ (entity.minimum_height / 2) - (loop.index * (entity.minimum_height / (loop.length + 1)))}}mm)$);
\draw[anchor=east] node at ($ (entity.west)+(-9mm,{{ (entity.minimum_height / 2) - (loop.index * (entity.minimum_height / (loop.length + 1)))}}mm)$){ {{input.name}} };




\draw[->] ($ (entity.east) + (0mm,{{ (entity.minimum_height / 2) - (loop.index * (entity.minimum_height / (loop.length + 1)))}}mm)$) -- ($ (entity.east) + (10mm,{{ (entity.minimum_height / 2) - (loop.index * (entity.minimum_height / (loop.length + 1)))}}mm)$);
\draw[anchor=west] node at ($ (entity.east) + (9mm,{{ (entity.minimum_height / 2) - (loop.index * (entity.minimum_height / (loop.length + 1)))}}mm)$){ {{ output.name }} };



\draw node at ($ (entity) - (0,0mm)$){ {{ entity.name }} };

\end{tikzpicture}
\end{center}

Ver\"andern Sie die Datei "`{{ entity.name }}.vhdl"' nicht!\\

---
insert your description of the task for the user here
---\\

Dieses Verhalten muss in der angeh\"angten Datei "`{{ entity.name }}\_beh.vhdl"' programmiert werden.



Um Ihre L\"osung abzugeben, senden Sie ein E-Mail mit dem Betreff "`Result Task {{tasknr_placeholder}}"' an {{submissionemail_placeholder}} und h\"angen Ihre Datei(en) an:

\begin{itemize}

\item ``{{ entity.name }}\_beh.vhdl"

\end{itemize}

\vspace{0.7cm}
Viel Erfolg und m\"oge die Macht mit Ihnen sein.

\end{document}

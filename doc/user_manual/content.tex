\section{Introduction}

TODO: some intro blabla


In the following, we distinguish between two parts of the configuration.
While the first part in Section \ref{sec:confautosub} deals with the configuration
of the Autosub framework itself, the second part in Section \ref{sec:writingtasks}
explains how to implement new tasks -- from the description to the tests -- so they
can be used by Autosub.


\cleardoublepage
\section{Autosub Configuration}\label{sec:confautosub}

The configuration of the Autsub framework is -- again -- comprised of two parts:

\begin{itemize}
\item The configuration file is described in Section \ref{sec:conffile} and can
is used to configure the framework. Things like the mail address and mail server
that shall be used can be configured here.
\item Autosub has some messages that are sent on events that not part of the
process of sending responses to new tasks. These messages can be {\it ``personalized``}.
Section \ref{sec:personal} explains, how this is done.
\end{itemize}

\subsection{The Configuration File}\label{sec:conffile}

\begin{verbatim}
[imapserver]
servername: imap.gmail.com
serverport: 587
username: fhcw.submission@gmail.com
password: nopasswd
email: fhcw.submission@gmail.com

[smtpserver]
servername: smtp.gmail.com
username: fhcw.submission@gmail.com
password: nopasswd
email: fhcw.submission@gmail.com

[general]
num_workers: 3

[challenge]
num_tasks: 3
\end{verbatim}

\subsection{Personalizing Autosub}\label{sec:personal}

\begin{description}
\item [welcome.txt]
\item [usage.txt]
\item [question.txt]  
\end{description}



\cleardoublepage
\section{Writing Tasks}\label{sec:writingtasks}

Implementing new tasks for Autosub is rather simple, there are just a few basic
naming conventions that have to be followed, in order to allow the Autosub
framework to correctly handle tasks.

These naming conventions are not very complicated, there are only four elements:

\begin{description}
\item [description.txt:] Contains a textual task description. For most tasks this will
be sufficient. The text in description.txt is sent to the user as the body of the e-mail
that contains a new task description that is sent after the last task was solved
\footnote{The notable exception is the first task that is sent after the users first e-mail
that subscribed him to Autosub.}.
\item [attachments:] For those cases, where a textual task description is not sufficient,
one or more attachments may be added to the e-mail that is sent to the user. For these cases
all you have to do, is to create a directory named {\it attachments}, and put all the files
you want to attach to the task description e-mail into that directory.
\item [tests.sh:] The Autosub framework itself does not know how to test the tasks. In order
to allow a simple implementation of the tests that check the hand-ins, we chose to use a
shell script as a basis. This has a number of advantages: Shell scripts are rather easy to
understand, so even if you do not know how to write them you can learn with a rather small
effort to use those couple of constructs you need to use them. At least to the point were
you are able to call a program written in your favorite programming language from that Shell
script. Furthermore, the use of a shell script removes the requirement of knowing the tests
that shall be run on the Autosub framework itself.
Another big advantage of using the shell script like this is, that the tests for each task
can be tested independently of the Autosub framework.
The rules are simple, just write a Shell script called {\it tests.sh} and let it return a 0
if all went well and a 1 if one of your tests failed. Depending on that, Autosub will decide
what to do -- if a 0 is returned it will send out the next task to the user and if a 1 is
returned, it will send an error message that describes what part of the submitted results
was wrong.
\item [error\_msg:] As described above, the {\it tests.sh} script may signal the Autosub
framework that submitted result was tested, and the test failed. In this case an error message
is needed that describes what kind of failure happened, and what kind of problem remains
to be fixed by the student, so he may go on to the next task. This is done using a file
called {\it error\_msg}. So the tests just write their error message into that file, and the
Autosub framework checks, whether such a file is available. If so, the content is included
into the body of the submission rejection mail.
\end{description}

The above describes already how a task is implemented. To clear the last ambiguities, let's
look at some example tasks now.

\subsection{Example Task: Hello World}

\cleardoublepage
\section{The first start of Autosub}



\subsection{Files Created by Autosub}

The following files are created by Autosub, when it is executed. If you run into troubles,
they will give you information on where to look for problems, and how to solve them.

\begin{description}
\item [autosub.log]
\item [autosub.db]
\item [autosub.pid]

\end{description}

\section{Introduction}

The following document specifies E-Learning platform for a \gls{vhdl}. This platform will first of all be used for the lectures Mikrocomputer and Digitale Integrierte Systeme. The main goal is to provide exercises to 
the students, that can be done from home, checked immediately and decrease the efforts of lecture (actually the lecturer will have an O(1) instead of an O(N)).

An important point is, that the tasks should be different for every student -- a basic version of the
example will be provided along with some parameters that are subject to change. These parameters are
then randomized in order to generate as many different examples as possible.

For the students, the interface shall be a very well known one -- it is based on E-Mail (which can be
assumed to be known to all students). This also has the advantage, that no usernames/passwords have to
be given to the students, only an e-mail address they can use to get examples from and send solutions to.
All possible interactions that can be performed by students are described in Section \ref{interactions}.

The system itself is specified in Section \ref{sec:autosub}.

NOTE: in the following {\it Structured Analysis} will be used to specify the system. If you are not familiar with this method, you can find details e.g. in \cite{demarco, gooma, cooling}.

\section{Interaction with the system}\label{interactions}
The VELS E-Learning system has 2 distinctive user groups: course operators and students. 
To satisfy the use cases for each of the 2 user groups the following interfaces have
to be defined:
\begin{itemize}
\item VELS Email Interface
\item VELS Web Interface
\item VELS direct server access
\end{itemize} 

The different use cases for students and operator and the responsible interface can be seen in 
Table \ref{tab:usestudent} and Table \ref{tab:useoperator}.

\begin{table}[h]
\centering
\begin{tabular}{||l | l||} 
    \hline
    Use Case & Responsible interface \\ [0.5ex] 
    \hline\hline
    register with the system & VELS Email Interface
    \\
    \hline
    get status in course & VELS Email Interface 
    \\
    \hline
    get current task & VELS Email Interface 
    \\
    \hline
    submit a task & VELS Email Interface
    \\
    \hline
    ask a question & VELS Email Interface
    \\
    \hline
\end{tabular}
\caption{Use cases for students}
\label{tab:usestudent}
\end{table}


\begin{table}[h]
\centering
\begin{tabular}{||c | c||} 
    \hline
    Use Case & Responsible interface \\ [0.5ex] 
     \hline\hline
    configure a course & VELS Web Interface
    \\
    \hline
    modify task generation or testbench generation & VELS direct server access
    \\
    \hline
    create a new task & VELS direct server access
    \\
    \hline
    view the progress for students & VELS Web Interface
    \\
    \hline
    view task statistics & VELS Web Interface
    \\
    \hline
    read log files & VELS direct server access 
    \\
    \hline
\end{tabular}
\caption{Use cases for course operators}
\label{tab:useoperator}
\end{table}

\newpage

\subsection{VELS Email Interface}\label{emailinterface}
The VELS Email Interface is the primary interface for students. Students are uniquely identified by their email 
address to interact with this interface. The student has to interact with a single, non-changing email address
during the whole duration of a course. This email address should contain the student id ("Matrikelnummer")
therefore only
the generic University of Technology email address can be used by the students. These email addresses have to be added to the system's whitelist by a course operator in order to enable them to be authorized for interaction.

The different actions a student can take are defined via email subject and address. The following cases
are possible:
\begin{itemize}
\item \textit{Sender email address is not on whitelist:} The student is sent an e-mail that he should use
    his generic University of Technology email address to interact with the system. If that does not help, he 
    should contact on of the course operators.
\item \textit{Sender email address is on whitelist, arbitrary Subject:} If the student is not registered 
    with the system, registration will be initiated. Registration with the system creates the appropriate
    data entries in the database for the student. If adding the new user was successful, two e-mails are 
    sent to the user: a welcome message with general information on how the system works and if already
    available the description of the first task. If the student is already registered he is sent an 
    e-mail with the usage for this interface. 
\item \textit{Subject contains the word "Question":} The student has a question about the course, this 
    question is forwarded to all course operators. The student receives a confirmation that the 
    question has been passed on.
\item \textit{Subject contains the phrase "Question Task N" where N is the task number:} The student 
    has a question about the course, this question is forwarded to all task operators for task N.
    The student receives a confirmation that the question has been passed on.
\item \textit{Subject contains the phrase "Result Task N" where N is the task number:} The student wants 
    to hand in a solution, for Task N. Students are allowed to hand in solutions for all tasks that 
    have already been solved previously, as well as the one task that has not been solved yet -- they
    are not allowed to hand in solutions for tasks they did not receive the task description yet. The
    solution is tested on the server. If the task solution was appropriate the student will receive the
    next task (or a congratulations message, if no more examples are available). If the tests were not
    successful, the student receives an error message that gives a hint on what may be wrong (output of the
    simulation tool, expected output vs. actual output, etc.)
\item \textit{Subject contains the word "Status":} The student is send an e-mail with the list of
    his completed tasks and his current task description.
\item \textit{\textit{default:}} The student is sent an e-mail with the usage for this interface.
\end{itemize}


\subsection{VELS Web Interface}\label{webinterface}

The VELS Web Interface is a configuration and status tool for the course operators.

Course operators can create a course. A course consists of a task queue, which explicitly orders a 
selection of the available tasks. The following actions for course configuration shall be implemented:
\begin{itemize}
\item Configuration of the task queue (position of task in the queue, start and end date for the task,
    points for task completion.
\item Configure course registration deadline
\item Modify course registration email whitelist 
\item Set Number of tasks 
\item Modify Archive Directory
\item Set Administrator E-Mail
\end{itemize}

Course operators can view the progress of each student and is presented with the following informations:
\begin{itemize}
\item Student name and student identification number ("Matrikelnummer").
\item Task the student is working on and the tasks the student has completed.
\item Score which was accumulated by the student.
\item Number of submissions for each task the student has received.
\item List of attached files for each task the student has received.
\end{itemize}

Course operators can view statistical informations about his course:
\begin{itemize}
\item Number of wrong and right submissions for each tasks.
\item General statistical informations about his course (e.g number emails received/sent).
\end{itemize}

The VELS Web Interface may also be used by the course operator to read the results at the end of the semester,
that is the points that have been scored by the individual students.


\subsection{VELS direct server access}\label{directserveraccess}
The VELS direct server access can be used by course operators to generate new task, modify code for task
generation, description and testing of existing tasks, read log files or fix bugs of the VELS system. 
It also can be used to view every submission a student has done. These procedures will be described in 
the VELS Usermanual.
\section{Appendix} \label{appendix}

\subsection{Config file fields} \label{app:config}
The configuration file can include the following grops and fields. Bold fields
are mandatory, all other fields can be ommited. Default paths are in respect to
the {\tt autosub/src} directory

{\bf [imapserver]}\\
\begin{tabular}{|p{2.5cm}|p{8cm}|p{2.5cm}|}
\hline
{\bf Field} & {\bf Description} & {\bf Default}\\
\hline
\hline
\textbf{servername} & Hostname.domain of the IMAP server. & ~\\
\hline
\textbf{username} & Username to be used to login. & ~ \\
\hline
\textbf{password} & Password to be used to login. & ~ \\
\hline
\textbf{email} & E-Mail address to be used at the IMAP server. & ~ \\
\hline
security & Security protocol to be used when connecting.
    Possible values: none ssl starttls & ssl \\
\hline
serverport & Port to be used. & ssl:993 else:143\\
\hline
\end{tabular}


{\bf [smtpserver]}\\
\begin{tabular}{|p{2.5cm}|p{8cm}|p{2.5cm}|}
\hline
{\bf Field} & {\bf Description} & {\bf Default}\\
\hline
\hline
\textbf{servername} & Hostname.domain of the SMTP server. & ~ \\
\hline
\textbf{username }& Username to be used to login. & ~ \\
\hline
\textbf{password} & Password to be used to login. & ~ \\
\hline
\textbf{email} & E-Mail address to be used at the SMTP server. & ~ \\
\hline
security & Security protocol to be used when connecting.
    Possible values: none ssl starttls & starttls\\
\hline
serverport & Port to be used. & ssl:465 starttls:587 none:25 \\
\hline
\end{tabular}

{\bf [system]}\\
\begin{tabular}{|p{2.5cm}|p{8cm}|p{2.5cm}|}
\hline
{\bf Field} & {\bf Description} & {\bf Default}\\
\hline
\hline
\textbf{num\_workers} & Number of Worker threads. This influences how many tests can conducted 
	in parallel. & ~ \\
\hline
\textbf{queue\_size} & Size of the thread communication queues. & ~\\
\hline
poll\_period & Period in seconds at which the mailbox is checked. & 60\\
\hline
semesterdb & The name and path of the semester database. & semester.db\\
\hline
coursedb &  The name and path of the course database. & course.db\\
\hline
log\_dir & Directory to put tasks and autosub log files. & logs  \\
\hline
\end{tabular}
\begin{tabular}{|p{2.5cm}|p{8cm}|p{2.5cm}|}
\hline
log\_threshhold & Threshhold from which level up log entries should be logged. 
	Possible values: DEBUG INFO WARNING ERROR & INFO\\
\hline
\end{tabular}

{\bf [course]}\\
\begin{tabular}{|p{2.5cm}|p{8cm}|p{2.5cm}|}
\hline
{\bf Field} & {\bf Description} & {\bf Default}\\
\hline
\hline
specialmsgs\_dir & Directory for messages that are sent to users on special events. & 
	SpecialMessages \\
\hline
\textbf{tasks\_dir} & Directory of the tasks implementations. & ~ \\ 
\hline
course\_name & The name of the course. & No name set \\
\hline
mode & The mode in which to run. Possible values: exam normal & normal \\ 
\hline
allow\_skipping & Allows users to skip ahead to the next task with a skip email. 
	Possible values: yes no & no \\
\hline
auto\_advance & Decides if users get auto advanced to a task which is activated by its 
	TaskStart. Possible values: yes no & no \\
\hline
allow\_requests & Decides if the users can request tasks by number. If this is activated 
	auto\_advance and allow\_skipping are set to no. Possible values: yes no & no \\ 
\hline
\end{tabular}

\subsection{Description of the message queues} \label{app:queues}

\begin{description}
\item [job\_queue] The job\_queue is used to trigger Worker threads to start testing 
	solutions that were received. The messages in the job\_queue are comprised of 
	the following fields:
    \begin{itemize}
        \item {\bf user\_id:} The unique User ID of the user who submitted this solution.
        \item {\bf user\_email:} The E-Mail address of the user who submitted this solution.
        \item {\bf task\_nr:} Number of the task that solution has been submitted for.
        \item {\bf message\_id:} The unique message-id of the E-Mail on the server.
    \end{itemize}

\item [sender\_queue] If a thread wants to send an E-Mail to a user, the Sender thread 
	is notified via this queue. The data needed by the Sender thread is in the messages 
	in the sender\_queue. These messages consist of the following fields:
    \begin{itemize}
        \item {\bf user\_id:} The unique User ID of the user who shall receive this E-Mail.
        \item {\bf recipient} E-Mail address of the recipient (the content of the 
			'To' field in the E-Mail).
        \item {\bf message\_type} The message type is used to decide on how to format 
			the E-Mail, and whether or not additional E-Mails have to be sent out 
			(e.g. when a question is handled). Possible message\_types are:

            \begin{itemize}
				\item {\tt Task} -- A task description.
				\item {\tt Success} -- A Task has been tested successfully.
				\item {\tt Failed} -- An error message for a failed test-run of a result 
					submission.
				\item {\tt SecAlert} -- Scanning the code revealed that this might be an 
					attack on the system.
				\item {\tt TaskAlert} -- An error message for failures in files that were 
					created for tasks.
				\item {\tt InvalidTask} -- A submission or request for a non-existent task.
				\item {\tt Usage} -- An E-Mail with usage explanation shall be sent to the 
					student. 
				\item {\tt Question} -- Confirm that a question was received.
				\item {\tt QFwd} -- Forward a question to the administrator.
				\item {\tt Welcome} -- Send a welcome message to a new student.
				\item {\tt NotAllowed} -- A user who is not on the whitelist sent a mail 
					to the system.
				\item {\tt SkipNotPossible} -- A user wants to skip to the next
					task, when he is either at the last task or the next task has not 
					started yet.
				\item {\tt TaskNotSubmittable} -- A user wants to submit to a task
					which has not started yet.
				\item {\tt TaskNotActive} -- An action concerning a task that is not active 
					yet.
            \end{itemize}

        \item {\bf task\_nr} Number of the task that message concerns (e.g. the current 
			one if the test failed, the next one if the task description shall be sent).
        \item {\bf message\_id:} The unique message-id of the E-Mail on the server.
    \end{itemize}

\item [logger\_queue] Trigger the logger to write a log message about an event that happened.
    \begin{itemize}
        \item {\bf message:} The text that describes the event that shall be logged.
        \item {\bf level:} The log-level of this log message; available log-levels 
			are DEBUG, INFO, WARNING and ERROR.
        \item {\bf src:} Name of the thread that reported the event that shall be logged.
		\item {\bf dst:} Log destination (autosub log, task message, task error).
    \end{itemize}

\item [generator\_queue] The generator\_queue is used to trigger the Generator thread 
	to generate a new (variant of) a task. That means that certain parameters of the 
	task are randomized, in order to assure that each student receives his/her very 
	own example. 
	\begin{itemize}
        \item {\bf user\_id:} The unique UserID of the User for whom the task is generated.
        \item {\bf user\_email:} E-Mail address of the User for whom the task is generated.
        \item {\bf task\_nr:} Number of the task that shall be generated.
        \item {\bf message\_id:} The unique message-id of the E-Mail on the server.
    \end{itemize}

\item [archive\_queue] The archive\_queue is used to announce that a E-Mail has been finished
	processing. This triggers archiving and removing task submissions from the active queue.
	own example. 
	\begin{itemize}
        \item {\bf message\_id:} The unique message-id of the E-Mail on the server.
        \item {\bf is\_finished\_job:} Flag to show that the message-id belongs to a finished 
			processed submitted task.
    \end{itemize}
\end{description}

\subsection{Semester database semester.db} \label{app:semester.db}

The database semester.db contains the following tables (and entries in those tables):

\begin{tabular}{|p{3cm}|p{10cm}|}
\hline
Table Name & Users \\
\hline
Description & Used to collect all necessary information on the Students.\\
\hline
Table Entries & \begin{itemize}
        \item {\bf UserId}: A unique UserID given at registration time (when the first E-Mail
            is received from the users E-Mail address).
        \item {\bf Name}: The name of the user as specified in the "from" field of the E-Mail.
        \item {\bf Email}: The E-Mail address of the user as specified in the "from" field of the
            E-Mail.
        \item {\bf FirstMail}: Timestamp of the first E-Mail received by this user.
        \item {\bf LastDone}: Timestamp of the email that contained the successful solution of the
            last task (only if this user has already finished the last task).
        \item {\bf CurrentTask}: The task the user is currently working on.
        \end{itemize} \\
\hline
\end{tabular}

\begin{tabular}{|p{3cm}|p{10cm}|}
\hline
Table Name & TaskStats \\
\hline
Description & Contains one entry for each available task, collecting statistics on the individual tasks.\\
\hline
Table Entries & \begin{itemize}
        \item {\bf TaskId}: Unique ID of the task.
        \item {\bf NrSubmissions}: Number of solutions received for the task with this TaskId.
        \item {\bf NrSuccessful}: Number of correct solutions received for the task with this TaskId.
        \end{itemize} \\
\hline
\end{tabular}

\begin{tabular}{|p{3cm}|p{10cm}|}
\hline
Table Name & TaskCounters \\
\hline
Description & Implements counters for certain events, examples for such events are: E-Mail received, E-Mail sent, 
question received, new user.\\
\hline
Table Entries & \begin{itemize}
        \item {\bf CounterId}: Unique ID of the counter.
        \item {\bf Name}: Name of the counter.
        \item {\bf Value}: Value of the counter.
        \end{itemize} \\
\hline
\end{tabular}

\begin{tabular}{|p{3cm}|p{10cm}|}
\hline
Table Name & UserTasks \\
\hline
Description & Used to map configured tasks to users -- e.g. store the generated examples so they can be 
    fetched later on for testing.\\
\hline
Table Entries & \begin{itemize}
    \item {\bf TaskNr}: Unique number of the task.
    \item {\bf UserId}: Unique ID of the user -- the combination of TaskNr and UserId make the entry unique.
    \item {\bf TaskParameters}: Either the parameters that describe the setting for this particular student, or a value that can be
        used to derive the parameters from.
    \item {\bf TaskDescription}: Message that describes the task that was specifically generated for the student.
        and shall be sent to the student.
    \item {\bf TaskAttachments}: List of attachments that shall be sent to the student (path+filename).
    \item {\bf NrSubmissions}: Number of submissions the student has done for this task.
    \item {\bf FirstSuccessful}: Number of the first successful submission.
\end{itemize} \\
\hline
\end{tabular}

\begin{tabular}{|p{3cm}|p{10cm}|}
\hline
Table Name & UserWhiteList \\
\hline
Description & A Whitelist of E-Mail addresses that shall be authorized to interact with the system.\\
\hline
Table Entries & \begin{itemize}
        \item {\bf UniqueId}: Unique ID in the whitelist table.
        \item {\bf Email}: E-mail address that shall be authorized to interact with the system.
		\item {\bf Name}: The name the user shall be assigned instead of the value from the
			"From" field of the registration E-Mail.
        \end{itemize} \\
\hline
\end{tabular}

\subsection{Course database course.db} \label{app:course.db}

The database course.db contains the following tables (and entries in those tables):

\begin{tabular}{|p{3cm}|p{10cm}|}
\hline
Table Name & TaskConfiguration \\
\hline
Description & Used to configure tasks, including their order, the scripts used to test 
	submissions, etc.\\
\hline
Table Entries & \begin{itemize}
    \item {\bf TaskNr}: The unique number of the task -- this number is used to establish 
		the order of tasks as received by the students.
    \item {\bf TaskStart}: Timestamp of when the task shall be available for students
		(if any). 
    \item {\bf TaskDeadline}: The deadline until which the task has to be successfully 
		submitted (if any).
    \item {\bf TaskName}: The name of the task folder of te task in respect to the 
	configured tasks\_directory.
	\item {\bf CommonFile}: Name of a common script that offers a tester functionalities.
    \item {\bf GeneratorExecutable}: The executable (script, binary, etc.) used to 
		generate unique examples for each student. If this is not set, all students will
		receive the same task.
    \item {\bf TestExecutable}: The executable (script, binary, etc.) used to test the 
		results submitted by the students.
    \item {\bf Score:} The points the student scores by solving this task.
    \item {\bf TaskOperator:} The E-Mail address of the course operator, who is 
		responsible for this task.
    \end{itemize}
    
\\
\hline
\end{tabular}

\begin{tabular}{|p{2.5cm}|p{11cm}|}
\hline
Table Name & SpecialMessages \\
\hline
Description & A collection of texts that shall be sent, in the case special events. \\
\hline
Table Entries & \begin{itemize}
    \item {\bf EventName}: Currently the following events are implemented:
        \begin{itemize}
        \item WELCOME -- A welcome message that is sent upon first E-Mail from a student 
			and that gives an explanation of how the system works.
        \item USAGE -- In case an E-Mail that can not be interpreted is received, this 
		messageshall be sent to the user.
        \item QUESTION -- A message that is sent as a confirmation if a question was 
			received.
        \item INVALID -- A message that is sent, in case a result for an invalid task has
                been received.
        \item CONGRATS -- A message that congratulates the student upon solving the 
			last task.
        \item REGOVER -- A message sent in case a student tries to register after the 
			registration deadline has passed.
        \item NOTALLOWED -- A mesage sent in case the student is not on the whitelist.
        \item CURLAST -- A mesage sent to inform the student that he has solved the 
			current last task.
        \item DEADTASK -- A mesage sent to inform the student that he submitted a task 
			which's deadline is allready over.
		\item SKIPNOTPOSSIBLE -- A message sent to the user that the skip he
			requested was not possible. Only sent if skipping is allowed.
		\item TASKNOTSUBMITTABLE -- A message sent to the user to tell him that
			he cannot submit a solution to a task he has not received yet.
        \end{itemize}
        \item {\bf Text}: The text that shall be sent in case of the event {\bf EventName}
			occurs.
    \end{itemize} \\
\hline
\end{tabular}

\begin{tabular}{|p{3cm}|p{10cm}|}
\hline
Table Name & GeneralConfig \\
\hline
Description & Store some general configuration for the semester to be used in the
VELS\_WEB interface. \\
\hline
Table Entries & \begin{itemize}
    \item {\bf ConfigItem}: Name of the Configuration Item.
    \item {\bf Content}: Content of the Configuration Item.
    \end{itemize} \\
\hline
\end{tabular}

\subsection{Handling of concurrent testing}
Handling of testing multiple submissions from users is handled in autosub in the following ways:
\begin{itemize}
\item Multiple Worker threads and seperated user tasks directories make it possible that
	multiple task submissions can be processed at the same time.
\item Active task submissions are put in an active queue and removed once they have been fully
	processed and an response has been sent to the student. Active task submissions that are 
	not processed after 5 minutes are presumed dead and removed from the active queue.
\item To prevent conflicts only one combination of user and task\_nr is allowed to be active at any
	time. If the user submits a solution for the same task before it has been fully processed, it 
	will be put in a backlog queue. Tasks from backlog are made active once the conflicting 
	submission has been fully processed.
\end{itemize}

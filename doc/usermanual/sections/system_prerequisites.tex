\section{VELS Pre-Requisites} \label{system_prerequisites}

In the following, the installation of pre-requisites for autosub is documented for 
the two current debian versions. Wheezy is currently the old-stable which was used 
for development of autosub, and jessie is the current stable which has been used to 
test autosub as well.

\subsection{Debian 7 (Wheezy)}

The probably easiest step is to install standard debian packages:

\begin{verbatim}
apt-get install python3 python3-mock python3-pip python3-nose zip flip git
apt-get install gnat-4.6-base libgnat-4.6 texlive-latex-extra pgf graphviz
apt-get build-dep python-matplotlib
\end{verbatim}

Unfortunately there is a small bug in one of the latex packages that we use,
therefore you need to apply a patch to that package:

\begin{verbatim}
cd /usr/share/texlive/texmf-dist/tex/generic/pst-circ
patch < /path/to/autosub/doc/patch/pst-circ.tex.patch
\end{verbatim}

In case the pdfs you get are not beautiful as you are used to, it might have
happended that an update was applied over that patch and you need to re-apply it (sorry for that little mess).


Next we need to upgrade easy\_install for python3 to a newer version -- this
is the pre-requisite for the installation of the matplotlib for python3.

\begin{verbatim}
easy_install3 -U distribute
\end{verbatim}

And finally we may now install some libraries that are not (yet) available in
the debian package system via using pip:

\begin{verbatim}
pip-3.2 install graphviz
pip-3.2 install bitstring
pip-3.2 install matplotlib
\end{verbatim}

Last, we install a vhdl simulator. If you want to use ghdl you may download
the debian package from sourceforge \footnote{http://sourceforge.net/projects/ghdl-updates/files/Builds/ghdl-0.31/Debian/} and install it using dpkg:

\begin{verbatim}
dpkg -i Downloads/ghdl_0.31-2wheezy1_amd64.deb
\end{verbatim}

All the dependencies of that package have already been resolved by the above
list of packages installed with apt.

To use the Simulator of Xilinx's ISE: ISim, see Section \ref{ISE-install}

At this point you may run the test-suite, as explained in Section \ref{sub:testingvels}.
If those tests run through without complications, everything you need to run the VELS
E-Learning system has been installed successfully.

\subsection{Debian 8 (Jessie)}

For jessie everything is even easier as more libraries are available for python3 in the
apt package pool now:

\begin{verbatim}
apt-get install python3 python3-mock zip flip graphviz python3-pip
apt-get install python3-matplotlib python3-nose gnat-4.9-base libgnat-4.9
apt-get install zlib1g-dev texlive-latex-extra pgf git
\end{verbatim}

Unfortunately there is a small bug in one of the latex packages that we use,
therefore you need to apply a patch to that package:

\begin{verbatim}
cd /usr/share/texlive/texmf-dist/tex/generic/pst-circ
patch < /path/to/autosub/doc/patch/pst-circ.tex.patch
\end{verbatim}

In case the pdfs you get are not beautiful as you are used to, it might have
happended that an update was applied over that patch and you need to re-apply it (sorry for that little mess).

So now, only two more packages that need to be installed using pip:

\begin{verbatim}
pip3 install bitstring
pip3 install graphviz
\end{verbatim}

Last, we install a vhdl simulator. If you want to use ghdl there you can use the version from \footnote{http://sourceforge.net/projects/ghdl-updates/files/Builds/ghdl-0.33/debian/} as there are no .deb packages for jessie of the older ghdl version (and vice-versa).

\begin{verbatim}
dpkg -i ghdl_0.33-1jessie1_amd64.deb
\end{verbatim}

To use the Simulator of Xilinx's ISE: ISim, see Section \ref{ISE-install}

At this point you may run the test-suite, as explained in Section \ref{sub:testingvels}.
If those tests run through without complications, everything you need to run the VELS
E-Learning system has been installed successfully.

\subsection{Debian 9 (Stretch)}

For stretch the standard debian packages are available for installation by:

\begin{verbatim}
apt-get install python3 python3-mock zip flip graphviz python3-pip
apt-get install python3-matplotlib python3-nose gnat-6 libgnat-6 
apt-get install zlib1g-dev texlive-latex-extra pgf git python3-pygt5
\end{verbatim}

Unfortunately there is a small bug in one of the latex packages that we use,
therefore you need to apply a patch to that package:

\begin{verbatim}
cd /usr/share/texlive/texmf-dist/tex/generic/pst-circ
patch < /path/to/autosub/doc/patch/pst-circ.tex.patch
\end{verbatim}

In case the pdfs you get are not beautiful as you are used to, it might have
happended that an update was applied over that patch and you need to re-apply it (sorry for that little mess).

So now, only two more packages that need to be installed using pip:

\begin{verbatim}
pip3 install bitstring
pip3 install graphviz
\end{verbatim}

Last, we install a vhdl simulator. The simulator of Xilinx's ISE installation is recommended since stretch is the new version and the ghdl for stretch version has been still developing. 

To use the Simulator of Xilinx's ISE: ISim, see Section \ref{ISE-install}

At this point you may run the test-suite, as explained in Section \ref{sub:testingvels}.
If those tests run through without complications, everything you need to run the VELS
E-Learning system has been installed successfully.

\subsection{Instalation Notes for Xilinx ISE 14.7}\label{ISE-install}

An additional option, to using the open-source software ghdl as the VHDL Simulation tool for VELS, is to use the Simulator of Xilinx's ISE: ISim. The ISE (``Full Installer for Linux'') can be downloaded from Xilinx's official download page. It requires registration and licensing agreement, but there is no charge. 

For installing ISE to the default location /opt/Xilinx/ where the VELS system expects it to be, you need permission to write to this location. So we will use the user root to call the graphical installer. To do so we need to allow root to use the users X server. As a user run:

\begin{verbatim}
$ xhost +
\end{verbatim}

to allow root (or anyone) to connect to your users X server temporarily. Now as root start the graphical installer, located at the top of the extracted download, with

\begin{verbatim}
# ./xsetup
\end{verbatim}

In case that you are using the KDE desktop environment, you have to remove the \verb!QT_PLUGIN_PATH! environment variable before starting the graphical installer:

\begin{verbatim}
# unset QT_PLUGIN_PATH
\end{verbatim}

If you are running the VELS system on a machine without a graphical user interface, the best way we found to get Xilinx's ISE onto a remote debian server was to copy the entire installation directory after it has been installed on a non-headless machine.

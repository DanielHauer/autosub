\section{Creating a new task for the VELS system} \label{create_new_task}

The VELS system incorporates a tool to create a new task and a generalized method for testing the submitted code of a student. The goal of both is to keep the necessary effort for creating a new task to the minimum.

\subsection{VHDL Task Creator Wizard} \label{sub:vhdltaskcreator}
The VHDL Task Creator Wizard is a graphical user interface to set up a new task. To use the wizard you will need python3-qt5 and python3-jinja2 installed on your system:
\begin{verbatim}
apt-get install python3-pyqt5 python3-jinja2
\end{verbatim}
The tool itself can be found in the VELS system under the path \path{autosub/tasks/tools/vhdl_task_creator} and can be called by running \verb|run.sh|.
When started the wizard will guide you through a series of windows to set up your new task. In the first window you can enter:

\begin{itemize}
\item {\bf Task name:} The name of the task. Used as the \todo{Bisher wird kein neuer Ordner erstellt- waere das leicht zu implementieren?} name of the task directory and in the task description for the user, the testbench\_template, in the task.cfg file and in the comments of the scripts generateTestBench.py and generateTask.py.
\item {\bf Output directory:} The location where the task directory will be saved.
\item {\bf User entities:} The name of the entities the user has to write the behavior for. The user will get the declaration of an entity in a file with the name \mbox{[entity name].vhdl}. And the user will write the behavior for the entity in a file with the name \mbox{[entity name]\_beh.vhdl}
\item {\bf Extra files:} Additional VHDL packages to be analyzed and elaborated in the order named here. For example the gates task uses these extra files to provide the user with gates defined in IEEE 1164.
\item \todo{Umaendern in ``Simulation Timeout'' im Task Creation Tool?} {\bf Timeout(s): } Time until the server hard stops the simulation of the user entities. Normally the simulation should be stopped from within the testbench well before the simulation timeout.
\item {\bf Attach wavefile:} If the wavefile of the simulation shall be sent to the user.
\item {\bf Use task constraint script:} If a script shall be called to check the users entities before simulation. For example the pwm task checks here if the user did use the wait statement instead of using the clock signal.
\end{itemize}

The ``next'' button brings you to a window where you can configure the inputs and outputs of the users entity. If you have defined more than one user entity then they can be configured in a following window. When adding the inputs and outputs for the entity you can set the parameters:

\begin{itemize}
\item {\bf Signal name:} The name of the signal. Any name conforming to the rules of the VHDL standard can be used here.
\item {\bf Signal type:} The available types are: std\_logic\_vector, std\_logic, signed, unsigned and custom. Dependant on the selected type are the available configurations. For example the std\_logic type leaves no configurations open except the signal name.
\item {\bf Length type:} The available options are: fixed, variable and single.
\item {\bf Length (Placeholder):} The bit-length of the selected signal type shall be entered here. If the length type is fixed then an integer greater 1 is expected. If the length type is variable then a string is expected as a placeholder.
\item {\bf Resulting definition:} A preview of the resulting definition is displayed here as the parameters get configured.
\end{itemize}

When all inputs and outputs are configured a summary for the to-be-created task is shown in the last window. When the ``finish'' button is pressed the task will be created in the output directory.

The created structure of the task directory is:
\begin{itemize}
\item {\bf templates:} Directory containing files which will be changed by the system for the task. These are the testbench\_template.vhdl and the task\_description\_template.tex. The task\_description\_template is used for creating the description which is sent to the user. The testbench\_template is used before starting a simulation to create a unique testbench. If the length of an entity input or output was chosen to be variable, then this directory will also include the file \mbox{[entity name]\_template.vhdl}.
\item {\bf static:} Directory containing files which will not be changed by the system itself. These are the files \mbox{[entity name].vhdl} and [entity name]\_beh.vhdl. The file \mbox{[entity name].vhdl} is the declaration of the entity for the user and \mbox{[entity name]\_beh.vhdl} is where the user shall write the behavior of the corresponding entity. If the length of an entity input or output was chosen to be variable, then this directory will not include the file \mbox{[entity name]\_beh.vhdl}.
\item {\bf scripts:} Directory containing python-scripts which generate the unique parameters of the task and fill out the task\_description\_template (generateTask.py) and which generate the testbench for the created task parameters (generateTestBench.py).
\item {\bf generator.sh} Script uses \LaTeX\,to create the task\_description PDF file for the user. Also attaches all needed files to the task description email for the user.
\item {\bf task.cfg:} File containing bash variables to adjust the generalized common\_tester.sh for the specific task.
\item {\bf tester.sh:} Generalized bash script which sources the parameters of the task.cfg file and calls the functions of the common tester.
\item {\bf description.txt:} Contains the text which is sent to the user in the task description email. \todo{description.txt is currently not generated by the wizard}
\end{itemize}


\subsection{Working with the common tester} \label{sub:testercommon}

The common tester provides a generalized method for testing the submitted code of a student. As mentioned one goal of this generalized testing method is to reduce the effort for creating a new task. Furthermore combining the testing for all tasks in one method allows to make future improvements at one centralized point.

When the VHDL Task Creator Wizard is used to create a new task (which is advised to do so) then, in the optimal situation, you will not have to directly deal with the common tester. When you are not using the Task Creator Wizard, or have to make further changes down the road, then the interface to work with the common tester is the task.cfg file. The task.cfg file is a configuration file for the common tester which is located in each task directory. It gets sourced by the tester.sh script. The tester.sh script also sources the common tester and the task constraint check if applicable and calls functions defined in the common tester. The task.cfg file includes the configurations:
\begin{itemize}
\item {\bf task\_name:} The variable task\_name is most importantly used in the common tester during elaboration and simulation of the submitted VHDL code. This is because the entity of the testbench is expected to be named \mbox{[task\_name]\_tb}.
\item {\bf userfiles:} The variable userfiles contains all the VHDL files expected to be submitted by the user. They are used to check if all files were submitted by the user and further on when the user VHDL files are analyzed.
\item {\bf entityfiles:} Variable containing the filenames of the entity files. The entity files contain the declaration of the user entities. These entity files are not from the user. The user writes the behavior of the entities in the user files.
\item {\bf extrafiles:} Variable containing the filenames of the extra entity files. Those extra VHDL entities are not from the user, but are supplied to the user in the task description. The extra files are sent to the user to provide ready coded VHDL entities the user can use in his task.
\item {\bf constraintfile:} The constraintfile variable contains the path to the constraint file in the task directory. The constraint file is sourced in the tester.sh script which is located in the task directory.
\item {\bf simulation\_timeout:} The simulation timeout is used during the simulation of the users VHDL code. If the simulation timeout is hit then the simulation process is killed.
\item {\bf attach\_wave\_file:} If set to `1' then the wavefile of the simulation will be attached in case the simulation of the users code was not successful.
\end{itemize}




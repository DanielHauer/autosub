\section{Setting up the VELS system} \label{setup}

\subsection{VELS server setup} \label{sub:serversetup}

The first thing to do when you want to install the autosub submission system and the VELS
web interface is to clone the git repository:

\begin{verbatim}
git clone https://github.com/andipla/autosub.git
\end{verbatim}

\subsection{Autosub submission config file}\label{sub:configfile}
The autosub config file is used to configure the connection to the mail server and set
initial configuration parameters for the system. It is usually located in the autosub root
folder {\tt "autosub/src/"} and is used when starting the autosub system. It is divided
in groups and entries. An example is given in the next subsection \ref{sub:exampleconfig}.

The configuration file should include the following entries:

\begin{tabular}{|p{2.5cm}|p{2.5cm}|p{8cm}|}
\hline
{\bf Group} & {\bf Field} & {\bf Description} \\
\hline
\hline
imapserver & servername & Hostname.domain of the IMAP server.\\
\hline
imapserver & serverport & Port to be used.\\
\hline
imapserver & security & Security protocol to be used when connecting 
    Possible values: none ssl starttls \\
\hline
imapserver & username & Username to be used to login. \\ 
\hline
imapserver & password & Password to be used to login. \\
\hline
imapserver & email & Email address at the IMAP server. \\
\hline
\hline
smtpserver & servername & Hostname.domain of the SMTP server.\\
\hline
smtpserver & serverport & Port to be used. \\
\hline
smtpserver & security & Security protocol to be used when connecting.
    Possible values: none ssl starttls \\
\hline
smtpserver & username & Username to be used to login. \\
\hline
smtpserver & password & Password to be used to login. \\
\hline
smtpserver & email & Email address at the SMTP server. \\
\hline
\hline
general & num\_workers & Number of Worker threads. This influences how many
	tests can conducted in parallel. \\
\hline
general & queue\_size & Size of the thread communication queues. \\
\hline
general & poll\_period & Period in seconds at which the mailbox is checked.\\
\hline
general & course\_name & The name of course.\\
\hline
general & semesterdb & The name and path of the database that shall be used for data specific to
    this semester. (default: semester.db) \\
\hline
general & coursedb & The name and path of the database that shall be used for data specific to
    this course. (default: course.db)\\
\hline
general & logfile & The name and path of the autosub logfile. (default: autosub.log)\\
\hline
general & auto\_advance & Decides if users get auto advanced to a task which is
    activated by its TaskStart. Possible values: yes no (default:no) \\
\hline
general & allow\_skipping & Allows users to skip ahead to the next task with a 
							 skip email. Possible values: yes no (default:no) \\
\hline
\hline
challenge & num\_tasks & Number of tasks in the course. \\
\hline
challenge & mode & The mode in which to run (exam or normal). \\ 
\hline
\end{tabular}

\newpage

\subsection{Example Configuration File} \label{sub:exampleconfig}

\begin{lstlisting}[frame=single,captionpos=b,caption=example.cfg, belowcaptionskip=4pt]]
[imapserver]
servername: imap.gmail.com
serverport: 993
security: ssl
username: submission@gmail.com
password: mysupersecurepassword
email: submission@gmail.com

[smtpserver]
servername: smtp.gmail.com
serverport: 587
security: starttls
username: submission@gmail.com
password: mysupersecurepassword
email: submission@gmail.com

[general]
num_workers: 3
queue_size: 200
poll_period: 60
course_name: Course Default
semesterdb: /tmp/semester.db
coursedb: /tmp/course.db
logfile: /home/andi/working_git/autosub/src/autosub.log

[challenge]
num_tasks: 6
mode: normal
\end{lstlisting}

\subsection{Starting the Daemon}

Autosub is implemented as a daemon process, that means all messages provided are written
to files (see Section \ref{logerror} for details on those files) -- nothing is written to
the console. The daemon is started using a shell-script located in {\tt autosub/src/}:

\begin{verbatim}
./autosub.sh start
\end{verbatim}

This starts the daemon using the default configuration file named {\tt default.cfg}. If you
wan to use your own configuration file, you have to pass the files name to the scrip when
starting the daemon:

\begin{verbatim}
./autosub.sh start myconfig.cfg
\end{verbatim}

To stop the daemon just run the command:

\begin{verbatim}
./autosub.sh stop
\end{verbatim}

\subsection{Setting up the VELS\_WEB Configuration Interface}
To use the VELS\_WEB Configuration Interface it has first to be installed and
configured using these steps:

Change to the VELS\_WEB directory.

\begin{verbatim}
python3 installer.py <pathtoautosub> <pathtoconfigfile>
\end{verbatim}
Use the same configfile you used for starting the autosub daemon! This step will
also download web2py to the user's home folder and set needed symbolic links to 
connect VELS\_WEB to the autosub system.

To use https you have to use a SSL key. The system expects the keyfiles
(server.crt , server.csr , server.key) to be in the web2py directory. To
generate keys and place them run the following (this can be skipped if you 
already have keyfiles you can use!):

\begin{verbatim}
./genkeys.sh    
\end{verbatim}

To start the VELS\_WEB daemon at port <port> and with password <password> run:
\begin{verbatim}
./daemon.sh start <port> <password> 
\end{verbatim}

After this step the VELS\_WEB interface will be reachable via your browser at
address:
\begin{verbatim}
https://<server_ip>:<port>/VELS_WEB
\end{verbatim}

To stop the daemon just run the command:

\begin{verbatim}
./daemon.sh stop
\end{verbatim}

\subsection{General Configuration}\label{sub:generalconfig}
Configuration items that can be changed dynamically are changeable in VELS\_WEB ->
General Config. These configurable items are:
\begin{itemize}
\item {\bf Number of tasks: }Can also be specified in the config file.
\item {\bf Registration Deadline: } Users who are try to register after this deadline will
    get an error e-mail.
\item {\bf Archive Directory: } Directory in which processed e-mails are moved, this
    directory has to be present on the IMAP server!
\item {\bf Administration Email: } Email addresses which get question and system error emails.

\end{itemize}

\subsection{Whitelisting} \label{sub:whitelisting}
Students that participate in the course have to be whitelisted in the VELS system. If the student
tries to write an e-mail to the system from an e-mail address that is not on the Whitelist, an E-Mail
with an error message is sent to him.

Whitelisting can be done in VELS\_WEB under the tab {\it Whitelist}. Email addresses can be added
one at a time or multiple at a time (mass subscription). Removal of a single e-mail address
can also be done in the VELS\_WEB.

\subsection{Configuring the Tasks} \label{sub:configTasks}
Existing tasks can be assembled into a task queue. This configuration is done in VELS\_WEB.
Each task in the queue has to be created with the following properties:
\begin{itemize}
\item {\bf TaskStart:} The start datetime for the task. The task will automatically
    be set to active if this datetime is reached. Users waiting for a task to become
    active will automatically receive an e-mail with the task description for that task.
\item {\bf TaskDeadline:} The end datetime for the task. Submissions for a tasks after
    this datetime will be rejected.
\item {\bf PathToTask:} The root path of the task, by default located in
    {\tt "autosub/src/tasks/implementation/VHDL"}, but that may be any arbitrary path.
\item {\bf GeneratorExecutable:} The path to the generator executable relative to the
    root path, by default the generator is located in the root path and named
    {\tt generator.sh}.
\item {\bf TestExecutable:} The path to the tester executable relative to the root path,
    by default the generator is located in the root path and named {\tt test.sh}.
\item {\bf Score:} The score a student gets for successful completion of the task. The
    scores for all completed tasks are added and can therefore also be used for grading.
\item {\bf TaskOperator:} The e-mail of the operator of the task. Currently this email
    is not used by the autosub system.
\item {\bf TaskActive:} The state of the task, for inactive tasks the generator won't
    be called.
\item {\bf Position:} The position of the task in the task queue. Do not change
    this during operation of your course!
\end{itemize}

\subsection{Notes on multiple VELS instances on the same machine}

The following should be adhered if you want to use multiple VELS instances on
one machine:
\begin{itemize}
\item Use different users for running the different instances. If you dont't do
	so you migth run into problems concerning the usage of the tmp directories of
	tasks in the test phase.
\item Be sure to use different ports when starting the VELS\_WEB daemon.
\end{itemize}

\subsection{Configuration Checklist} \label{sub:configChecklist}

\begin{enumerate}
\item Installed all needed libraries and tools for autosub, the tasks and VELS\_WEB.
\item Configured the email server, including an email archive folder.
\item Created a config file for autosub.
\item Started the autosub system via {\tt autosub.sh start <configfile>}.
\item Started VELS\_WEB using the daemon.
\item Configured all parameters in General Config in VELS\_WEB.
\item Configured all Tasks in VELS\_WEB.
\item Whitelisted all students in VELS\_WEB.
\end{enumerate}

If you forget one of this steps or mis-configure, autosub tries to generate a meaningful
message, still it's nicer to get everything running without being yelled at.

\subsection{The Exam Mode}\label{sub:exammode}
In Exam Mode the students additionally get sent a minimal testbench to test their design.
To enable test mode, change the challenge-mode to exam in VELS\_WEB -> General Config for
an existing course (databases exist) and in the config file for a new course (databases
don't exist).

The remaining configuration is similar to configuration for normal mode.

\subsection{Testing VELS}\label{sub:testingvels}

%QUESTION Martin: Can we write that semester.db and course.db have to be present?,
%some of my tests need them to be present..
% BUMP!!

VELS has a testing suite located in {\tt "autosub/src/tests"}. It consists of unit and
doctests, testing both the autosub system and the tasks itself. With this test suite you
can test if everything is set up the right way before starting a course.To run the test
suite issue the following command from the {\tt "autosub/src/"} directory:

\begin{verbatim}
nosetests3 --with-doctest --doctest-extension=txt --nologcapture -v
\end{verbatim}

This will run all doctest as well as unittest test cases.

If you also want the code coverage of the test-suite then run it as follows:

\begin{verbatim}
nosetests3 --with-doctest --doctest-extension=txt --nologcapture -v --with-coverage
\end{verbatim}

While the above commands run all of the tests, it is also possible to include or exclude only
specific tests. In example it is possible to only execute the load test:

\begin{verbatim}
nosetests3 --nologcapture -s -v  tests/load_test.py:Test_LoadTest
\end{verbatim}

or it is possible to exclude the load test (which makes sense, as that one takes a considerable
amount of time):

\begin{verbatim}
nosetests3 --nologcapture -s -v --ignore-files=load_test.py
\end{verbatim}

The following is tested by the testsuite:
\begin{itemize}
\item Connecting to the databases.
\item Logging a message.
\item Sending an email.
\item Functionality of the activator thread.
\item Functionality of the generator thread.
\item Functionality of the sender thread.
\item Functionality of the fetcher thread.
\item Functionality of the common used functions.
\item Behavior under high load (load test).
\item Creation off all description files for every task.
\item Compilation of all generated VHDL files with ghdl.
\item Tester functionalities for given right submissions with ghdl.
\end{itemize}


\section{Logging and Error Detection of the VELS System}\label{logerror}

The autosub system is implemented as a daemon process. As such it does not
directly print any information to the {\tt stdout} (i.e. the screen), instead
all messages for the administrator are stored in log files. All log and error files
are stored in the root autosub directory {\tt "autosub/src"}. Due to the modular
nature of autosub, not everything can be included into one big fat log file
(although most of it is), but some things are sent to separate files.
In essence autosub uses 3 files that will tell the administrator about everything
that is going on:

\begin{description}
\item [autosub.log] This is the main log file that is used by the actual autosub
daemon software to log everything. The format in this log file is as follows:

\begin{verbatim}
<date> <time> [<logger name>] <loglevel> <logmessage>
\end{verbatim}

Here is a small section of {\tt autosub.log} taken from startup -- first the daemon checks
whether a directory {\tt users} used to store the users submissions is existing, then
it checks if necessary tables in the database exist. After those checks have been
done further threads (as described in Section \ref{sub:autosub}) are started, and
announce themselves.

{\scriptsize
\begin{verbatim}
2015-09-17 22:23:40,629 [autosub.py  ] WARNING: Directory already exists: users
2015-09-17 22:23:40,630 [autosub.py  ] DEBUG: table SpecialMessages does not exist
2015-09-17 22:23:43,679 [autosub.py  ] DEBUG: table TaskConfiguration does not exist
2015-09-17 22:23:43,955 [autosub.py  ] DEBUG: table GeneralConfig does not exist
2015-09-17 22:23:45,878 [sender      ] INFO: Starting Mail Sender Thread!
2015-09-17 22:23:45,884 [fetcher     ] INFO: Starting Mail Fetcher Thread!
2015-09-17 22:23:45,884 [fetcher     ] DEBUG: Imapserver: 'mail.intern.tuwien.ac.at'
2015-09-17 22:23:45,884 [generator   ] INFO: Task Generator thread started
2015-09-17 22:23:45,890 [activator   ] INFO: Starting activator
2015-09-17 22:23:45,890 [Main        ] INFO: Used config-file: testzid.cfg
2015-09-17 22:23:45,891 [Worker1     ] INFO: Starting Worker1
2015-09-17 22:23:45,892 [Main        ] INFO: All threads started successfully
2015-09-17 22:23:45,892 [Worker1     ] INFO: Worker1: waiting for a new job.
2015-09-17 22:23:45,893 [Worker2     ] INFO: Starting Worker2
\end{verbatim}
}

As you can see from theses examples, a lot of information is available, with
exact timestamps and a way to know which thread the message is coming from.
In addition the debug level gives a sense of how important (or even critical)
the message is.

The log-level is currently hard-coded into the logger thread ({\tt logger.py}) and can only
be changed in the code. Log-file size can get several MB in size we do currently
see no need to set the log-level to something other than DEBUG and gather all information
that can be very valuable in case something goes wrong.

\item [autosub.stderr and autosub.stdout] As already explained in Section \ref{sub:autosub},
the autosub daemon calls generator and tester scripts for the individual tasks.
The output of these tasks can not easily be piped into the autosub logging thread.
If it could, there would also be no way to check if the script actually do it,
and the developer of a task could instead again just print to {\tt stdin} and
{\tt stderr}. Therefore the logging is simply done by piping the output of called
generator and tester scripts into these files.

This means, that all messages printed by the generator and tester executables are
piped into those two files and can be used to find out if the generator and / or
tester executable (or the programs they call) cause any problems.
\end{description}

While these three files give all information needed to debug problems, autosub tries
to give information on severe problems on all available channels. An example would be,
if no task is configured, the email sent to the student will contain the message that
something went wrong as well -- although the error can be seen in the {\tt autosub.log}
file as well.

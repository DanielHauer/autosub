\section{System Interaction} \label{system_interaction}
The VELS system is an E-Learning system for students who are learning the \gls{vhdl}. The
interaction between students and the system is solely via E-Mail. Configuration of the
system is done with a configuration file (for configuration items that can not change dur-
ing runtime) and a web interface (for configuration items that can change dynamically).

The VELS E-Learning system has 2 distinctive user groups: course operators and students. 
To satisfy the use cases for each of the 2 user groups the following interfaces have
been defined:
\begin{itemize}
\item VELS E-Mail Interface
\item VELS Web Interface
\item VELS Direct Server Access
\end{itemize} 

The different use cases for students and operator and the responsible interface can be 
seen in Table \ref{tab:usestudent} and Table \ref{tab:useoperator}.

\begin{table}[h]
\centering
\begin{tabular}{||l | l||} 
    \hline
    Use Case & Responsible interface \\ [0.5ex] 
    \hline\hline
    register with the system & VELS E-Mail Interface
    \\
    \hline
    get status in course & VELS E-Mail Interface 
    \\
    \hline
    request/get a task & VELS E-Mail Interface 
    \\
    \hline
    submit a task solution & VELS E-Mail Interface
    \\
    \hline
    ask a question & VELS E-Mail Interface
    \\
    \hline
\end{tabular}
\caption{Use cases for students}
\label{tab:usestudent}
\end{table}


\begin{table}[h]
\centering
\begin{tabular}{||l | l||} 
    \hline
    Use Case & Responsible interface \\ [0.5ex] 
    \hline\hline
    configure a course & VELS Web Interface
    \\
    \hline
    configure a task queue & VELS Web Interface
    \\
    \hline
    modify task generation or testbench generation & VELS Direct Server Access
    \\
    \hline
    create a new task & VELS Direct Server Access
    \\
    \hline
    view the progress for students & VELS Web Interface
    \\
    \hline
    view task statistics & VELS Web Interface
    \\
    \hline
    read log files & VELS Direct Server Access 
    \\
    \hline
\end{tabular}
\caption{Use cases for course operators}
\label{tab:useoperator}
\end{table}

\newpage

\subsection{VELS E-Mail Interface}\label{emailinterface}
The VELS E-Mail Interface is the primary interface for students. Students are uniquely 
identified by their E-Mail address to interact with this interface. The student has to 
interact with a single, non-changing E-Mail address during the whole duration of a course. 
These E-Mail addresses have to be added to the system's whitelist by a course operator 
in order to enable them to be authorized for interaction. If a user that is not on the 
whitelist sends an E-Mail to the system, the system replies with a message to use the 
whitelisted address or contact a course operator.

The different actions a student can take are defined via E-Mail subject and address. The 
following cases when using a whitelisted E-Mail address and system reactions are implemented:
\begin{itemize}
\item \textit{Arbitrary Subject, user not registered:} Registration is initiated. Registration with the 
	system creates the appropriate data entries in the database for the student. If adding the new user was 
	successful, the user gets an welcome E-Mail with general information on how the system works.
\item \textit{Subject contains the word "Question":} The student has a question about the course, this 
	question is forwarded to all course operators. The student receives a confirmation that the 
    question has been passed on.
\item \textit{Subject contains the phrase "Question Task N" where N is the task number:} The student 
    has a question about a specific task of the course, this question is forwarded to all task operators 
	for task N. The student receives a confirmation that the question has been passed on.
\item \textit{Subject contains the phrase "Result Task N" where N is the task number:} The student wants 
    to hand in a solution, for Task N. Students are allowed to hand in solutions for all tasks that 
    have already been solved previously, as well as the one task that has not been solved yet -- they
    are not allowed to hand in solutions for tasks they did not receive the task description yet. The
    solution is tested on the server. If the task solution was appropriate the student will receive the
    next task (or a congratulations message, if no more examples are available). If the tests were not
    successful, the student receives an error message that gives a hint on what may be wrong (output of the
    simulation tool, expected output vs. actual output, etc.)
\item \textit{Subject contains the word "Status":} The student is send an E-Mail with the list of
    his completed tasks and his current task description.
\item \textit{\textit{default:}} The student is sent an E-Mail with the usage for this interface.
\end{itemize}

\newpage 

Depending on the chosen task queue mode additional keywords are accepted (for a closer description of different modes 
look at Section \ref{sub:task_queues}):

\textbf{With request mode:}
\begin{itemize}
\item \textit{Subject contains the word "Request Task N":} The student wants to request the task N. If the task can be 
	given to the student at this moment, the student will receive an E-Mail containing all informations on the task.
\end{itemize}

\subsection{VELS Web Interface}\label{webinterface}

The VELS Web Interface is a configuration and status tool for the course operators.

Course operators can create a course. A course consists of a task queue which is a selection of the available tasks.
The following actions for course configuration  are implemented:
\begin{itemize}
\item Configuration of the task queue
\item Configure course registration deadline
\item Modify course registration E-Mail whitelist 
\item Modify archive directory
\item Set administrator E-Mail
\end{itemize}

Course operators can view the progress of each student and is presented with the following informations:
\begin{itemize}
\item Student name.
\item Task the student is working on and the tasks the student has completed.
\item Score which was accumulated by the student.
\item Number of submissions for each task the student has received.
\item List of attached files for each task the student has received.
\end{itemize}

Course operators can view statistical informations about their course:
\begin{itemize}
\item Number of wrong and right submissions for each task.
\item General statistical informations about the course (e.g number E-Mails received/sent).
\end{itemize}

The VELS Web Interface may also be used by the course operator to read the results at the end of the semester,
that is the points that have been scored by the individual students. In depth description on how to use the 
VELS\_WEB interface can be found in Section \ref{VELS_WEB} and Section \ref{system_setup}.


\subsection{VELS Direct Server Access}\label{directserveraccess}
The VELS Direct Server Access is used by course operators to  modify code for task
generation, description and testing of existing tasks, read log files or fix bugs of the VELS system. 
It also can be used to view every submission a student has done. 

\documentclass[a4paper,12pt]{article}
\usepackage{a4wide}
\usepackage{tikz}
\usetikzlibrary{calc}

\begin{document}
\pagestyle{empty}
\setlength{\parindent}{0em} 
\section*{RAM (Data Memory)}

Your task is to program the behavior of an entity called "RAM". This entity is declared in the attached file "RAM.vhdl" and has the following properties:
\begin{itemize}
\item Input:  Clk with type std\_logic
\item Input:  en\_read%%enReadIndex with type std\_logic
%%ENREAD\item Input:  en\_read2 with type std\_logic
\item Input:  en\_write%%enWriteIndex with type std\_logic
%%ENWRITE\item Input:  en\_write2 with type std\_logic
\item Input:  input%%inIndex with type std\_logic\_vector
%%IN2\item Input:  input2 with type std\_logic\_vector
\item Input:  addr1 with type std\_logic\_vector
\item Input:  addr2 with type std\_logic\_vector
\item Output: output%%outIndex with type std\_logic\_vector
%%OUT2\item Output: output2 with type std\_logic\_vector
\end{itemize}

\begin{center}
\begin{tikzpicture}
\draw node [draw,rectangle, minimum height=45mm, minimum width=35mm,rounded corners=2mm,thick](entity){};
\draw[->,thick] ($ (entity.west)-(10mm,18mm)$) -- ($ (entity.west) - (0mm,18mm)$);
\draw node at ($ (entity.west)-(15mm,18mm)$){Clk};

%%ENREAD\draw[->,thick] ($ (entity.west)-(10mm,13mm)$) -- ($ (entity.west) - (0mm,13mm)$);
%%ENREAD\draw node at ($ (entity.west)-(19mm,13mm)$){en\_read2};
\draw[->,thick] ($ (entity.west)-(10mm,9mm)$) -- ($ (entity.west) - (0mm,9mm)$);
\draw node at ($ (entity.west)-(19mm,9mm)$){en\_read%%enReadIndex};
%%ENWRITE\draw[->,thick] ($ (entity.west)-(10mm,5mm)$) -- ($ (entity.west) - (0mm,5mm)$);
%%ENWRITE\draw node at ($ (entity.west)-(19mm,5mm)$){en\_write2};
\draw[->,thick] ($ (entity.west)-(10mm,1mm)$) -- ($ (entity.west) - (0mm,1mm)$);
\draw node at ($ (entity.west)-(19mm,1mm)$){en\_write%%enWriteIndex};

%%IN2\draw[->,thick] ($ (entity.west)-(10mm,-5mm)$) -- ($ (entity.west) - (0mm,-5mm)$);
%%IN2\draw node at ($ (entity.west)-(17mm,-5mm)$){input2};
\draw[->,thick] ($ (entity.west)-(10mm,-9mm)$) -- ($ (entity.west) - (0mm,-9mm)$);
\draw node at ($ (entity.west)-(17mm,-9mm)$){input%%inIndex};

\draw[->,thick] ($ (entity.west)-(10mm,-15mm)$) -- ($ (entity.west) - (0mm,-15mm)$);
\draw node at ($ (entity.west)-(17mm,-15mm)$){addr2};
\draw[->,thick] ($ (entity.west)-(10mm,-19mm)$) -- ($ (entity.west) - (0mm,-19mm)$);
\draw node at ($ (entity.west)-(17mm,-19mm)$){addr1};

\draw[->,thick] ($ (entity.east) + (0mm,-2.5mm)$) -- ($ (entity.east) + (10mm,-2.5mm)$);
\draw node at ($ (entity.east) + (19mm,-2.5mm)$){output%%outIndex};
%%OUT2\draw[->,thick] ($ (entity.east) + (0mm,2.5mm)$) -- ($ (entity.east) + (10mm,2.5mm)$);
%%OUT2\draw node at ($ (entity.east) + (19mm,2.5mm)$){output2};

\draw node at ($ (entity) - (0,14mm)$){RAM};

\end{tikzpicture}
\end{center}

Do not change the file "RAM.vhdl".
\\

The "RAM" entity shall behave according to the following RAM:
\\

Outputs of the entity and also the content of the RAM can only change on rising edges of the clock cycle. The initial content of the memory is zero.\\
The address of a memory location, that the data is going to be written to or read from, is located on the address lines. When the reading or writing enables are active, the processes of reading or writing are performed on the rising edge of the clock cycle.

\begin{itemize}
%%Desc0\item The first address is the address of first writing operation and also the reading operation and the second address is for the second writing operation only. 
%%Desc0\item When en\_read is '1', the content of addr1 is read.
%%Desc0\item When en\_write1 is '1', input1 is written to addr1.
%%Desc0\item When en\_write2 is '1', input2 is written to addr2.
%%Desc0\item Reading and second writing do not happen to the same address at the same time.
%%Desc0\item en\_read and en\_write1 are not active at the same time.
%%Desc0\item Two writing operations can happen at the same time to different addresses only.

%%Desc1\item The first address is the address of first reading operation and also the writing operation and the second address is for the second reading operation only. 
%%Desc1\item When en\_read1 is '1', the content of addr1 is read.
%%Desc1\item When en\_read2 is '1', the content of addr2 is read.
%%Desc1\item When en\_write is '1', input is written to addr1.
%%Desc1\item Reading and writing operations do not happen to the same address at the same time.
%%Desc1\item en\_read1 and en\_write are not active at the same time.

%%Desc2\item The first address is the address of first reading and also the first writing operations and the second address is for the second reading and writing operations only. 
%%Desc2\item When en\_read1 is '1', the content of addr1 is read.
%%Desc2\item When en\_read2 is '1', the content of addr2 is read.
%%Desc2\item When en\_write1 is '1', input1 is written to addr1.
%%Desc2\item When en\_write2 is '1', input2 is written to addr2.
%%Desc2\item Reading and writing operations do not happen to the same address at the same time.
%%Desc2\item en\_read1 and en\_write1 are not active at the same time.
%%Desc2\item en\_read2 and en\_write2 are not active at the same time.
%%Desc2\item Two writing operations can happen at the same time to different addresses only.

%%Desc3\item The first address is the address of writing operation and the second address is for the reading operations only. 
%%Desc3\item Reading and writing operations can be done at the same time from/to the same address. Priority is with the reading operation.
%%Desc3\item When en\_write is '1', input1 is written to addr1.
%%Desc3\item When en\_read is '1', the content of addr2 is read.
\item When the memory is not read, a number of "Z" with the same length as the output is displayed on the output.

\end{itemize}

 The length of the address is %%ADDRLENGTH bits, the length of input data %%WRITELENGTH, the length of output is %%READLENGTH, and the length of each location of the memory is %%DATASIZE as well.
\begin{itemize}
%%Desc4\item When en\_write is active, the input data is written to the specified address. The length of the input is as the same as each memory location.
%%Desc5\item The length of input is as twice as each memory location. Therefore, the less significant half of input should be written to the specified address and the more significant half of data should be written to the next address.
%%Desc6\item When en\_read is active, the data is read from the specified address. The length of the output is as the same as each memory location.
%%Desc7\item The length of output is as twice as each memory location. Therefore, the content of specified address is read and put in the less significant half of output and the content of next address is read and transferred to the more significant half of output.
\end{itemize}

This behavior has to be programmed in the attached file "RAM\_beh.vhdl".
\\

To turn in your solution, write an email to %%SUBMISSIONEMAIL with Subject "Result Task %%TASKNR" and attach your file "RAM\_beh.vhdl".

\vspace{0.7cm}
Good Luck and May the Force be with you.


\end{document}
\grid